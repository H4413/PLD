\section{Introduction}

L'aide au choix de la solution est présent dans le rapport de choix et le powerpoint
de présentation final.

\section{Bilan Moral}

\subsection{Outils}

Les outils ont pu être jugés comme peu intuitif, il est nécessaire de s'adapter 
au formalisme ARIS découvert en 3IF. Néanmoins on peut s'adapter. Il est en revanche
 dommageable de ne pas avoir pu utiliser SAP au contraire de certains groupes. 

Les outils de gestion de projet, comme Redmine initialement prévu, ont été progressivement
 abandonnés. Le principal problème étant la lourdeur de ce dispositif pour un projet 
"scolaire". Nous n'avions pas de budget alloué sur ce projet comme une réelle entreprise.
Il était donc fastidieux de "reporter" le temps sur des tâches unitaires.
La saisie des temps étaient faite sur le tableau de bord.

\subsection{Affectif}

Ce projet est un projet de conseil, que l'on peut réaliser chez des SSII 
comme Accenture ou Atos par exemple. Il ne peut pas satisfaire l'ensemble des profils
présents au département informatique. Néanmoins en tant que chef de projet j'ai 
essayé d'adapter les rôles en fonction des profils de chacun. En effet le PLD 
requiert un large panel de compétence tant au chef de projet qu'au reste de son
équipe.

Ce projet nous a laissé une grande liberté sur notre organisation. 
Il a été l'occasion d'expérimenter différentes méthodes et de faire un benchmark 
organisationnel sur le long terme. Il est très formateur sur la gestion de projet ou
la gestion des conflits.

\subsection{Améliorations continues}

Après un rapide brainstorming et une réflexion sur l'ensemble des étapes faites 
par le projet nous proposons les points suivants : 

\begin{itemize}
\item Exemples GSTP sur ARIS plus visible et test de SAP\\ 
Cette remarque a été reprise par Mme Legait sur la suppression d'un TD par rapport à 
l'année dernière. Il serait intéressant de remettre en place cette séance de TD 
pour découvrir la réalité de SAP. Pour les contraintes de calendrier, nous 
proposons d'instaurer une séance intermédiaire supplémentaire...
\item Revue et Présentation intermédiaire \\
Il apparait important de rendre plus concret la spécification fonctionelle. 
En effet une critique apparue de M. Amghar était le manque de rapport entre 
les solutions et les axes améliorations. Cette présentation pourrait être ainsi 
une opportunité pour l'ensemble de l'équipe de bien comprendre l'ensemble du système
 avant de partir sur les deux solutions.
\item Interventions de professionnel du secteur\\
Nous avons suggéré de faire la présentation finale avec un industriel du secteur. 
Nous avions proposé Altran, parrain de la future promotion de 4IF. 
Il semble que la piste SPIE soit explorer actuellement.
\item La mise en place d'interview métier apparaît indispensable. \\
L'un des principaux reproches que j'ai entendu est : "Ce projet n'est pas concret". 
Il est fondé et justifié. En effet nous basons l'ensemble de notre étude sur un 
rapport de 15 pages et quelques questions posés à nos chers enseignants 
changeant de casquette. L'opportunité de faire des interviews métiers 
permettrait de rendre plus concret la découverte du métier de GSTP. \\
On peut y mettre plusieurs formes comme faire un entretien d'une demi-heure avec 
des vrais questions métiers, le groupe devra préparer cet entretien. La seconde
méthode serait d'enregistrer des interview comme le suggérer M. Amghar.
\end{itemize}

\section{Bilan général}

\subsection{Charges}
\subsubsection{Charge Effective}
\def\colorrow{\rowcolor[gray]{0.8}}
\begin{longtable}{|c|c|c|c|c|c|c|c|c|c|c|c|c|}
\hline
&{\bf Total}&{\bf S50}&{\bf S1}&{\bf S2}&{\bf S3}&{\bf S4}&{\bf S5}&{\bf S6}&{\bf S7}&{\bf S8}&{\bf S9}&{\bf S10}\\
\hline
\endhead
\colorrow {\bf Karen}&43,5&3&3,5&5,5&3,5&4,5&4,5&4&5&4&0&6\\
\hline
{\bf Victor}&42,5&3&3,5&3,5&3,5&5&4,5&3,5&5&5&0&6\\
\hline
\colorrow {\bf Clément}&45,5&3,5&3,5&5,5&4,5&5,5&4,5&5,5&5&4&0&4\\
\hline
{\bf Raphaël}&46&3,5&3,5&6,5&4&4,5&3,5&3,5&5&4&0&8\\
\hline
\colorrow {\bf Hugo}&50&3,5&3,5&3,5&5&7,5&4,5&6,5&6&4&0&6\\
\hline
{\bf Quentin}&59,5&3,5&5&8&4,5&7,5&7&5&4&4&1&10\\
\hline
\hline
\colorrow {\bf Total}&287&20&22,5&32,5&25&34,5&28,5&28&30&25&1&40\\
\hline
{\bf Différentiel}&287&20&22,5&32,5&25&34,5&28,5&28&30&25&1&40\\
\hline
\end{longtable}


\subsubsection{Charge prévue}
\def\colorrow{\rowcolor[gray]{0.8}}
\begin{longtable}{|c|c|c|c|c|c|c|c|c|c|c|c|c|}
\hline
&{\bf Total}&{\bf S50}&{\bf S1}&{\bf S2}&{\bf S3}&{\bf S4}&{\bf S5}&{\bf S6}&{\bf S7}&{\bf S8}&{\bf S9}&{\bf S10}\\
\hline
\endhead
\colorrow {\bf Karen}&45,5&3&4&5&6&3,5&4,5&4,5&4,5&4,5&2&4\\
\hline
{\bf Victor}&39,5&3&4&4&5&3,5&3,5&3,5&3,5&3,5&2&4\\
\hline
\colorrow {\bf Clément}&44&3,5&4&5&5&4,5&3,5&3,5&3,5&5,5&2&4\\
\hline
{\bf Raphaël}&51&3,5&4&4&5&5&5&4&5,5&4&7&4\\
\hline
\colorrow {\bf Hugo}&46&3,5&4&4&5&3,5&3,5&4,5&4,5&6,5&3&4\\
\hline
{\bf Quentin}&60,5&3,5&4&9&6&5&4,5&3,5&5,5&6,5&9&4\\
\hline
\hline
\colorrow {\bf Total}&286,5&20&24&31&32&25&24,5&23,5&27&30,5&25&24\\
\hline
\end{longtable}


\subsubsection{Analyse}
La charge pouvait être difficile à évaluer pour un projet de cette ampleur.
Par chance, le chiffre total était relativement bon. 

Néanmoins, les enseigner à tirer peuvent être les suivants : 
\begin{itemize}
\item Le mode projet est vraiment présent. L'approche de l'échéance final nous
a permi d'intensifier notre productivité et notre qualité de réalisation. Il est
malheureusement difficile de le prévoir au début de projet.
\item La charge a été relativement homogène au contraire de la productivité. Elle
est difficilement quantifiable en étant étudiant. En effet nous ne pouvons pas
"vendre" des prestations d'étudiants à différents tarifs.
\item Le phasage ne pouvait pas être finement découpé par manque d'expérience
\end{itemize}

\subsection{Retards}
Karen a été régulierement en retard. Elle a eu un retard sur 50\% des séances. 
Donc 4 avec plus d'un quart d'heure. Dans l'ensemble les autres éléments ont été
 ponctuels.

\subsection{Motivation}
Des indicateurs de suivi de motivation ont été mis en place, mais l'équipe n'a 
pas joué le jeu pour donner des valeurs sincères. 
Cet indicateur est présent dans le tableau de bord final.

\subsection{Risques}
Les risques prévus ont été des facteurs problématiques. 

\section{Bilans personnels}
