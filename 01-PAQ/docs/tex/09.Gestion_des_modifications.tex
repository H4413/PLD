\section{Gestions des modifications}

Les modifications mineures sont suivies par le versionnement.\\
Il est obligatoire de mettre un message
expliquant les modifications à chaque ajout.\\
Pour ce faire il faut livrer ses modifications avec la commande
\begin{verbatim}
 git commit nomfichier -m"modification fichier nomfichier: ajout XXX"
\end{verbatim}
Lors de modifications importantes (ex: fin de rédaction d'une
sous-partie), il faut ajouter le nom de l'auteur
et le nouveau numéro de version au tableau de suivi du document.
Celui ci se nomme «00.b.1.suivi\_modifications.tex».


Le tableau de bord utilise les correspondances suivantes pour quantifier
l'avancement:
\begin{description}
    \item[0\%:]  Pas commencé.
    \item[20\%:] Démarré.
    \item[50\%:] En cours, bon avancement.
    \item[70\%:] Candidat à la validation. 
    \item[80\%:] Fini, nécessite seulement des petites modifications,
                    relu mais non validé.
    \item[90\%:] Candidat à la vérification RQ. 
    \item[100\%:]Document réputé comme fini.
\end{description}
