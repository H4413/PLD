\section{Méthodes, Outils et Règles}

\subsection{Méthodes}
Les points suivants des méthodes AGILE sont appliqués:
\begin{itemize}
\item Réunions en début de séance
\item Réunions d'avancement en fin de séance
\item Suivi de projet en ligne (Redmine)
\item Travail collaboratif lors des séances
\end{itemize}

\subsection{Outils}
L'équipe utilise Redmine pour la gestion de projet.
\label{outils}

L'outil de versionnement Git est utilisé pour gérer les différentes versions
des livrables. Notre dépôt est hébergé par \url{www.GitHub.com}.

Les documents sont rédigés avec le langage de composition \LaTeX.
L'outil Make est utilisé, en coordination avec un script bash, pour générer
chaque livrable à partir des sources.

\subsection{Règles et normes}

Un livrable est organisé, sur le serveur, sous forme d'arbre: à la racine
se trouvent le makefile, un fichier «infos» contenant les informations sur le
livrable, ainsi que le répertoire «docs» contenant: un script de génération
du livrable «gen.sh», un répertoire «tex» où sont stockés les fichiers source \LaTeX,
et un répertoire «pics» pour mettre les images.

Dans le répertoire «tex», on trouve un fichier par partie, nommé de la façon suivante:
\textsl{XX.nom\_partie.tex} avec \textsl{XX} le numéro de la partie.
Les noms de fichiers ne doivent contenir ni accents ni espaces, et les fichiers
sont encodés de préferences en unicode.

Les fichiers commençant par «00» concernent la première page, la gestion de suivi
et la table des matières.
