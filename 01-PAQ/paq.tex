% Plan d'assurance qualité - H4413

\documentclass[twoside]{article}
\usepackage{hyperref}


\usepackage{graphicx}
\usepackage{subfig}
\usepackage{placeins}


% Unicode encoding  
\usepackage[utf8x]{inputenc}


% Colorfull Text
\usepackage{xcolor}


% \euro
\usepackage{eurosym}


% Language settings:
\usepackage[french]{babel}

\usepackage[T1]{fontenc}


% Tables
\usepackage{array}
\usepackage{longtable}


% Hyperrefferences  
\usepackage{hyperref}


\title{Plan d'assurance qualité}
\author{H4413}
% Page layout settings
\usepackage{geometry}
\geometry{
	a4paper,  % 21 x 29,7 cm
	body={160mm,240mm},
	left=30mm, 
	top=25mm,
	headheight=7mm, 
	headsep=4mm,
	marginparsep=4mm,
	marginparwidth=27mm
}


% Spacing:
\usepackage{setspace}


% Headers and footers:
\usepackage{fancyhdr}
\pagestyle{fancy}
          \fancyhf{}
          \fancyfoot[LE,RO]{\textcolor[gray]{0.3}{\thepage}}
          % Rulers width
          \renewcommand{\footrulewidth}{.3pt}
          \renewcommand{\headrulewidth}{.0pt}
\fancyfoot[LO,RE]{\textcolor[gray]{0.3}{H4413}}
\fancyfoot[CO,CE]{\textcolor[gray]{0.3}{Plan d'assurance qualité}}


% Vars & functs
% Paths
\newcommand\PIXPATH{./docs/pics}
\newcommand\SRCPATH{./docs/src}

% Object:
\newcommand\Object{PAQ}

% End of line(forced):
\newcommand\el{\hfill\\}

% Lists design:
\renewcommand{\labelitemi}{$\diamond$}
\renewcommand{\labelenumii}{\arabic{enumi}.\arabic{enumii}}


% Begining of the document
\begin{document}

	%Including all the files:

    % Fichier ./docs/tex/00.a.premiere_page.tex

% Front Page 

% Title:
\maketitle

\thispagestyle{empty}

\hfill\\
\vfill

% Picture

\begin{center}
    \includegraphics[width=5cm]{\PIXPATH/frontPage}
\end{center}

\section*{Objet}
\Object

    % Fichier ./docs/tex/00.b.1.suivi_modifications.tex

% Suivi du document

% Modifications
\section*{Modifications du document}

\begin{center}
\begin{longtable}{|m{14mm}|m{36mm}|m{36mm}|m{60mm}|}
\hline
Version & Auteur & Date & Modification\endhead \hline
% Version
0
& % Auteur
Raphaël Lizé
& % Date
7 janvier 2011
& % Modification
Création
\\\hline
% Version
0.1
& % Auteur
Hugo PASTORE DE CRISTOFARO
& % Date
16 janvier 2011
& % Modification
Relecture et complétion
\\\hline
% Version
0.2
& % Auteur
Raphaël Lizé
& % Date
16 janvier 2011
& % Modification
Relecture et complétion
\\\hline
% Version
1
& % Auteur
Raphaël Lizé
& % Date
17 janvier 2011
& % Modification
Complétion après vérification
\\\hline
% Version
1.1
& % Auteur
Raphaël Lizé
& % Date
20 janvier 2011
& % Modification
Modifications mineures
\\\hline
% Version
1.2
& % Auteur
Raphaël Lizé
& % Date
21 janvier 2011
& % Modification
Ajout signification \% de progression
\\\hline
% Version
2
& % Auteur
Raphaël Lizé
& % Date
21 janvier
& % Modification
Dernières modifications
\\\hline
% Version

& % Auteur

& % Date

& % Modification

\\\hline
\end{longtable}
\end{center}


    % Fichier ./docs/tex/00.b.2.suivi_veva.tex

% Suivi du document
% Validations

\section*{Vérifications et validations du document}

\begin{center}
\begin{longtable}{|m{15mm}|m{36mm}|m{36mm}|m{60mm}|}
\hline
 & Responsable & Date & Remarques\endhead \hline
% Validé/vérifié par
Vérifié
& % Responsable
Quentin Villers
& % Date
17 janvier 2011
& % Remarques
à  étoffer
\\\hline
% Validé/vérifié par
Validé
& % Responsable
Quentin Villers
& % Date
21 janvier 2011
& % Remarques

\\\hline
% Validé/vérifié par

& % Responsable

& % Date

& % Remarques

\\\hline
\end{longtable}
\end{center}

\pagebreak

    % Fichier ./docs/tex/00.c.toc.tex

% Table of contents 
\tableofcontents
\vfill
\pagebreak

    % Fichier ./docs/tex/01.preliminaires.tex

% Préliminaires

%\section{Préliminaires}
%    \subsection{Cadre du PAQP}

%    \subsection{Logiciels concernés par le PAQP}

%    \subsection{Responsabilités associées au PAQP}

%    \subsection{Procédure d'évolution du PAQP}

%    \subsection{Procédure à  suivre en cas de non-application du PAQP}
%\pagebreak

    % Fichier ./docs/tex/02.introduction.tex

% Introduction
\section{Introduction}
Ce document précise les dispositions relatives à la qualité pour l'ensemble du projet.
%\pagebreak

    % Fichier ./docs/tex/03.documents_de_ref.tex

% Documents de référence et applicables

\section{Documents de référence et applicables}
    
    \subsection{Documents de référence}
\begin{itemize}
   \item Dictionnaire de la langue française
   \item Livrables fournis par le client
   \item Polycopiés R. Aubry
\end{itemize}

    \subsection{Documents applicables}
        Plan d'Assurance Qualité.

    % Fichier ./docs/tex/04.terminologies.tex

% Terminologie et applications

%\section{Terminologie et applications}
%Application

%\vfill
%\pagebreak

    % Fichier ./docs/tex/05.organisation_humaine_comite_pilotage.tex

% Organisation humaine du comité de pilotage du projet

\section{Organisation humaine du comité de pilotage du projet}

Le comité de pilotage du projet est organisé de la façon suivante:

    \subsection{Rôle des différents intervenants}
\begin{description}
\item[Chef de projet:]M. Quentin Villers
\item[Responsable qualité:]M. Raphaël Lizé
\item[Responsable communication:]M. Hugo Pastore
\item[Experts métiers:]Mlle Karen Abento, MM. Clément Geiger et Victor Borges
\end{description}

    \subsection{Relations entre les intervenants}
\begin{description}
\item[Clients:]MM. Youssef Amghar, Mohamed Ouhalima et Pierre-Alain Millet,
Mme Anne Legait.
\end{description}

    \subsection{Planning des séances et règles}
Les séances sont organisées et animées par le chef de projet et se
déroulent de cette façon:
\begin{enumerate}
\item initialisation de la séance, 15 minutes:
    \begin{itemize}
        \item retours,
        \item objectifs du jour,
        \item assignation des tâches;
    \end{itemize}
\item revue éventuelle;
\item travail sur le projet;
\item clôture.
\end{enumerate}

Les collaborateurs suivront ces règles:
\begin{itemize}
\item être ponctuel,
\item entretenir des relations professionnelles les uns envers les
        autres en envers les clients.
\item Les entreprises s'engagent à répondre sous un jour ouvré pour 
        permettre aux équipes d'analyser le plus fidèlement possible
        le travail des intervenants.
\item Tenir les collaborateurs au courant de l'avancée du travail.
\item Répondre aux mails sous 1 jour ouvré.
\end{itemize}

    % Fichier ./docs/tex/06.qualite_niveau_processus.tex

% Qualité au niveau du processus

%\section{Qualité au niveau du processus}
%
%    \subsection{Présentation de la démarche de développement au niveau Projet}
%
%        \subsubsection{Généralités}
%
%        \subsubsection{Phase d’étude préalable}
%
%        \subsubsection{Phase d’étude détaillée}
%
%        \subsubsection{Phase d’intégration système}
%
%        \subsubsection{Phase de validation système}
%
%        \subsubsection{Phase de mise en \oe{}uvre sur site pilote}
%
%    \subsection{Règles de qualité pour l’ingénierie concurrente}
%
%        \subsubsection{Règles sur la rédaction d’un cahier des charges d’un sous-projet}
%
%        \subsubsection{Règles sur la définition précise des résultats attendus pour chaque sous-projets}
%
%        \subsubsection{Règles sur le suivi qualité des sous-projets}
%
%        \subsubsection{Règles sur la définition de critères d’acceptation des sous-projets avant intégration}
%
%    \subsection{Présentation des démarche de développement au niveau sous-projets}
%
%        \subsubsection{Liste des processus de développement susceptibles d’être retenus pour le
%            développement des sous-projets}
%
%        \subsubsection{Description du cycle de développement n\degre 1}
%
%            \par{Liste des étapes}
%
%            \par{Étape numéro 1}
%
%            \begin{description}
%                \item[Documents en entrée]
%                \item[Documents en sortie]
%                \item[Conditions de validation de l'étape%]
%            \end{description}
%
%            \par{Suivi de projet}
%\pagebreak

    % Fichier ./docs/tex/07.Documentation.tex

%\section{Documentation}
%
%\subsection{Structuration de la documentation}
%La documentation est strucuturée par livrable. Le dossier de
%documentation est placé sur un dépôt versionné.
%Chaque livrable est identifié par un numéro et est contenu dans 
%un dossier nommé de la manière suivante : XX.nom\_livrable.
%
%
% \subsection{Liste des documents de gestion de projet}
%
%
%\subsection{Liste des documents relatifs à la qualité}
%
%\begin{itemize}
%    \item Plan d'Assurance Qualité (ce document)
%\end{itemize}
%
%\subsection{Manuels d'utilisation et de mise en \oe{}uvre}



    % Fichier ./docs/tex/08.Gestion_de_configuration.tex

\section{Gestion de configuration}

\subsection{Conventions d'identification}
Les livrables possèdent des tableaux, en début de document, permettant
de tracer son historique: qui a modifié le document, quand, avec un
commentaire; qui a relu/validé le livrable, quand, avec un commentaire.

\subsubsection{Responsabilités}
Chacun est responsable de ses entrées dans les tableaux: les rédacteurs
pour le tableau des versions, le responsable qualité et le chef de projet
pour le tableau de validation.

Le responsable qualité vérifie ces entrées.

\subsubsection{Procédures de gestion de la configuration}
La gestion de configuration se fait avec l'outil \textsl{Git} (cf. \ref{outils}).

%Lorsqu'un collaborateur veut effectuer une 

\subsection{Gestion des ressources partagées}
Toutes les ressources sont déposées sur le dépôt de versionnement.

L'outil \textsl{ARIS} est utilisé avec un compte partagé.

    % Fichier ./docs/tex/09.Gestion_des_modifications.tex

\section{Gestions des modifications}

Les modifications mineures sont suivies par le versionnement.\\
Il est obligatoire de mettre un message
expliquant les modifications à chaque ajout.\\
Pour ce faire il faut livrer ses modifications avec la commande
\begin{verbatim}
 git commit nomfichier -m"modification fichier nomfichier: ajout XXX"
\end{verbatim}
Lors de modifications importantes (ex: fin de rédaction d'une
sous-partie), il faut ajouter le nom de l'auteur
et le nouveau numéro de version au tableau de suivi du document.
Celui ci se nomme «00.b.1.suivi\_modifications.tex».


Le tableau de bord utilise les correspondances suivantes pour quantifier
l'avancement:
\begin{description}
    \item[0\%:]  Pas commencé.
    \item[20\%:] Démarré.
    \item[50\%:] En cours, bon avancement.
    \item[70\%:] Candidat à la validation. 
    \item[80\%:] Fini, nécessite seulement des petites modifications,
                    relu mais non validé.
    \item[90\%:] Candidat à la vérification RQ. 
    \item[100\%:]Document réputé comme fini.
\end{description}

    % Fichier ./docs/tex/10.Methodes_outils_regles.tex

\section{Méthodes, Outils et Règles}

\subsection{Méthodes}
Les points suivants des méthodes AGILE sont appliqués:
\begin{itemize}
\item Réunions en début de séance
\item Réunions d'avancement en fin de séance
\item Suivi de projet en ligne (Redmine)
\item Travail collaboratif lors des séances
\end{itemize}

\subsection{Outils}
L'équipe utilise Redmine pour la gestion de projet.
\label{outils}

L'outil de versionnement Git est utilisé pour gérer les différentes versions
des livrables. Notre dépôt est hébergé par \url{www.GitHub.com}.

Les documents sont rédigés avec le langage de composition \LaTeX.
L'outil Make est utilisé, en coordination avec un script bash, pour générer
chaque livrable à partir des sources.

L'avancement du projet peut être suivi via un document partagé hébergé
sur le service \url{docs.google.com}.

La modélisation sera réalisée avec le logiciel \textsl{ARIS}.

\subsection{Règles et normes}

Un livrable est organisé, sur le serveur, sous forme d'arbre: à la racine
se trouvent le makefile, un fichier «infos» contenant les informations sur le
livrable, ainsi que le répertoire «docs» contenant: un script de génération
du livrable «gen.sh», un répertoire «tex» où sont stockés les fichiers source \LaTeX,
et un répertoire «pics» pour mettre les images.

Dans le répertoire «tex», on trouve un fichier par partie, nommé de la façon suivante:
\textsl{XX.nom\_partie.tex} avec \textsl{XX} le numéro de la partie.
Les noms de fichiers ne doivent contenir ni accents ni espaces, et les fichiers
sont encodés de préferences en unicode.

Les fichiers commençant par «00» concernent la première page, la gestion de suivi
et la table des matières.

    % Fichier ./docs/tex/11.Controle_des_fournisseurs.tex

%\section{Contrôle des Fournisseurs}
%
%\subsection{Exigences vis-à-vis des sous-traitants}
%Les sous-traitants sont tenus d'appliquer toutes les règles de ce plan d'assurance qualité.
%Ils peuvent créer leur propre plan d'assurance qualité qui doit strictement hériter du document présent.
%\subsection{Logiciels achetés, loués ou imposés}
%L'utilisation de Latex est la seule contrainte logicielle imposée aux sous-traitants.
%\vfill
%\pagebreak

    % Fichier ./docs/tex/12.Reproduction_protection_livraison.tex

\section{Reproduction, Protection, Livraison}

\subsection{Précautions à prendre lors de la reproduction}
Toute reproduction ne doit se faire que pour une copie locale d'un document.

\subsubsection{Protection des données contre les incidents}
Les données sont versionnées par Git. Le serveur GitHub garde des copies de toutes les versions précédentes.
A partir du moment où les données ont été mises en ligne, il est toujours possible de les récupérer.
C'est pourquoi il est conseillé de partager le travail régulièrement.

\subsection{Modalités de livraison}
Les livraisons se font sur moodle par le chef de projet.
Il est interdit à quelconque membre de l'équipe de livrer un document sans
validation du contenu par le chef de projet et de la forme par le resposable
quélité. 

\subsubsection{Délais}
Les délais sont inhérents aux conditions imposées par le client. Les délais imposés par le
planning du chef de projet en héritent et sont les dates butoires réelles à respecter. 


    % Fichier ./docs/tex/13.Suivi_application_PQ.tex

\section{Suivi de l'application du Plan Qualité}

\subsection{Principes}
Un outil de gestion de projet est mis en place. Il s'agit d'un Redmine.
La validation de la qualité se fait par ce biais.

\subsection{Interventions du responsable qualité sur la démarche de développement}
Le responsable qualité vérifie la forme de tous les livrables. Il vérifie aussi l'utilisation des outils mis en place
et l'application des méthodes choisies.

Le responsable qualité participe aussi à la gestion des risques sur le projet.

\subsection{Modalités de réception des résultats des sous-projets avant intégration}
Le Responsable Qualité valide les livrables indépendamment avant tout rendu.

\vfill
\pagebreak

    % Fichier ./docs/tex/14.Conclusion.tex

%\section{Conclusion}

%\vfill
%\pagebreak

% The end
\end{document}

