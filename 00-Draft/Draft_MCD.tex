% Doit compiler à partir de ce simple fichier
% Mettre NOM et DATE et FICHIER RELATIF
\title{Modification du modèle conceptuel de données}
\author{Hugo PASTORE DE CRISTOFARO}
\date{02/02/2011}

\documentclass[a4paper]{article}

\usepackage[top=2cm, bottom=2cm, left=2cm, right=2cm]{geometry}

\usepackage[utf8x]{inputenc}
\usepackage[french]{babel}

\begin{document}
\maketitle

\section{Introduction}
Peu de modifications du MCD, simple ajout d'une classe commande standard.

%Si le RQ est motivé il peut en faire un tableau....

\section{Ajout d'une classe Commande Standard}

La classe commade standard (STD_CMD) représente des listes de pièces de 
rechange standard à commander à chaque début de chantier en fonction de son type.

\subsubsection{Description de l'entité}
\begin{description}
    \item[CODE-LISTE] Idenifiant de la liste, TYPE : INT, LG : 10
    \item[NOM-LISTE] Nom de la liste de pièces, TYPE : STRING
    \item[DESC-LISTE] Description de la liste, TYPE : STRING
\end{description}

\subsubsection{Relations et cardinalité}
\begin{description}
    \item[COMM/LISTE] COMMANDE : Commande d'une liste standard (1,N)\el
        STD_CMD : liste standard (0,N)
    \item[LISTE/PR] STD_CMD : Liste des pièces de rechange (0,N)\el
        PIECE-RECHANGE :  Pièce de rechange.(O,N)
\end{description}


\subsection{}

\end{document}
