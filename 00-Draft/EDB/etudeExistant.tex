% Doit compiler à partir de ce simple fichier
% Mettre NOM et DATE et FICHIER RELATIF

\documentclass[a4paper]{article}

\usepackage[top=2cm, bottom=2cm, left=2cm, right=2cm]{geometry}

\usepackage[utf8]{inputenc}
\usepackage[french]{babel}

\title{Draft : Etude de l'existant\\Département Achat}
\author{Clément}
\date{14 janvier 2010}

\begin{document}
\maketitle

\section{Introduction}

L'étude de l'existant a pour but la réalisation d'un audit au niveau de la
direction impactée par le projet : la Direction du Matériel. Pour chaque
département, on synthétisera le fonctionnement en résumant les processus
majeurs, rappelant les objets métiers et décrivant l'informatique utilisée
(matériel et logiciel).


\section{Département achat}


\subsection{Processus majeurs}

Le département d'achat s'occupe principalement de l'achat de matériel et de
pièces de rechange ; il réalise également des achats de prestations en cas
de besoin et gère son catalogue de fournisseurs.

Achat matériel différent achat PR : prix, acte d'investissement autorisé
par direction ou fonctionnement normal, délai


\subsubsection{Achat de matériel}

La direction du matériel déclenche une demande d'achat de matériel auprès
de la direction générale lorsqu'elle constate une pénurie de matériel. La
DG effectue si nécesaire un arbitrage entre les différentes demandes.\\
Lorsqu'une demande est acceptée, elle est confiée au département achat et
la procédure ci-après est suivie :

\begin{enumerate}
\end{enumerate}

La DM prend en charge les demandes acceptées : dép achat

CHoix d'un fournisseur,
Négociation
Formalisation d'un contrat d'achat
Travail de suivi de la commande jusqu'à réception
Réception de commande, contrôle
Si problème, réserve : retour
Sinon : la commande est intégrée au parc


\subsubsection{Achat de pièces de rechanges}

Déclenchée par le magasinier
"Automatique" dans la plupart des cas, peut conduire aux phases de l'achat
de matériel parfois.


\subsubsection{Achat de prestations}

Principalement location de matériel auprès d'entreprises extérieures car
matériel indisponible chez nous (peu importe la raison)
Solution de réactivité, transparente pour le chantier ; augmenter la
précision des intervenants si connus

\begin{enumerate}
\item Commande 
\item Suivi de commande
\item Soldage de commande
\end{enumerate}

\subsubsection{Gestion des fournisseurs}

Le département achat doit maintenir à jour son catalogue de fournisseur.
Cela inclut les étapes suivantes :
\begin{enumerate}
\end{enumerate}

\subsection{Objets métiers}

La liste des objets métiers identifiée est la suivante :

\begin{itemize}
\item Direction du Matériel
\item Direction Générale
\item Département Achat
\item Magasinier
\item Fournisseur
\item Commande
\item Livraison
\item Matériel
\item Pièce de rechange
\item Prestation
\end{itemize}


\subsection{SI}

\subsection{Matériel}
Le département Matériel dispose de deux PC et deux imprimante (un couple
PC/Imprimante par employé du département).

\subsection{Logiciel}
Le département achat utilise les applications suivantes :
\begin{itemize}
\item Gestion des fournisseurs (environ 300 fournisseurs)
\item Gestion des bons de commande (2 à 3 achats de gros matériel par mois
et 600 achats de petits matériels par an).
\end{itemize}

Les applications ont été développées en interne.


\subsection{Conclusions : +/-}
\end{document}
