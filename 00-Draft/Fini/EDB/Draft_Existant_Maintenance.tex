% Doit compiler à partir de ce simple fichier
% Mettre NOM et DATE et FICHIER RELATIF
\title{Etude de l'existant - D\'epartement Maintenance}
\author{H4413}
\date{14-01-2011}

\documentclass[12pt]{article}

\usepackage[utf8]{inputenc}
\usepackage[french]{babel}

\begin{document}
\maketitle

\begin{abstract}
%Insérez ici les documents relatifs
\end{abstract}

\section{Département Maintenance}
Cette partie vise à étudier l'existant dans le département Maintenance et
identifier les dysfonctionnements.

\subsection{Processus}
Il y a 2 processus mis en place dans le département Maintenance, un pour chaque
service :
\subsubsection{Gestion des Pièces de Rechange}
Ce processus consiste à gérer l'approvisionnement, la réception, la 
valorisation et la gestion des pièces de rechange.
Le processus se résume comme suit :
\begin{enumerate}
\item Réception et Inventoriation d'une ou d'un ensemble de pièces de rechange
\item Calcul du stock
\item Calcul des besoins en pièces de rechange
\item Commande de pièces de rechange
\end{enumerate}

\subsubsection{Maintenance}
Ce service utilise une procédure de réponse à une demande de maintenance ou
traite une intervention planifiée périodique ou non. Il s'agit soit de 
réparation d'une panne soit d'une mesure préventive.
Le processus se résume de la manière suivante:
\begin{enumerate}
\item Identifier une opération de maintenance ou une panne
\item Lancer l'opération de maintenance
\item Récupérer une pièce de rechange le cas échéant
\item Réaliser l'opération de maintenance
\end{enumerate}

\subsection{Objets métier}
La liste des objets métier identifiés est la suivante :
\begin{itemize}
\item Employé maintenance
\item Magasin PR (Pièce de Rechange)
\item Inventaire PR
\item Atelier maintenance
\item Planning entretien
\item Opération de maintenance
\item Demande maintenance
\item Demande réapprovisionnement PR
\item Demande intervention pour une panne
\item Commande PR
\item Avis maintenance
\item Avis livraison
\end{itemize}

\subsection{Système Informatique}
Le département maintenance possède 2 ordinateurs et 2 imprimantes.
Les ateliers sur les chantiers utilisent les ordinateurs disponibles sur les
chantiers.
Les applications utilisées servent à la gestion du stock des pièces de rechange
ainsi qu'à la planification de la maintenance.

\subsection{Dysfonctionnements}
Les processus de maintenance sont corrects mais sont fortement ralentis par
la communication non formalisée par informatique des demandes et des avis.

L'absence d'application gérant les interventions non planifiées est le principal
dysfonctionnement identifié. Seule la planification de la maintenance est gérée
par informatique. De plus les chantiers n'ont pas accès au SI du siège. La 
procédure de demande d'intervention est donc soit très lente, soit faite par
l'atelier sur place avec un suivi papier. Dans tous les cas, la commande de 
pièces de rechange est ralentie et les tâches d'inventaire et de calcul des
stocks compliquées par l'absence de suivi informatique rapide.

\end{document}
