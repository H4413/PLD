% Doit compiler à partir de ce simple fichier
% Mettre NOM et DATE et FICHIER RELATIF

%Pensez à mettre le cas échéant si il s'agit d'un nouveau processus, d'un processus
%existant remodelé (+ mettre le fichier auquel il s'y rapporte)
%Mettre en forme en latex

\documentclass[a4paper]{article}

\usepackage[utf8x]{inputenc}

\title{Annotation du MCT - DRAFT}
\author{Cgg, Victor}
\date{6 février}

\usepackage[top=2cm, bottom=2cm, left=2cm, right=2cm]{geometry}

\usepackage[utf8x]{inputenc}
\usepackage[french]{babel}

\begin{document}
\maketitle

\begin{abstract}
Ce document regroupe les annotations effectuées sur les différents MCT
employés au sein de GSTP. Pour inclure ce travail au livrable final,
copier/coller le code \LaTeX suivant.\\
NB : le paragraphe "Commande de pièces de rechange" n'est pas mis en forme.
Il faut se plaindre auprès de \textbf{Victor} :-P.
\end{abstract}

\subsection{Commande de pièces de rechange}

Le responsable de chaque atelier fais une demande de pièce de rechange au
magasin le plus proche que sera délivré qu'une fois par semaine.  La
livraison est faite par "zone chantier", c'est-à-dire, un camion est
responsable pour desservir plusieurs ateliers que sont proches entre eux
(30 km, par exemple). Cela réduit les coûts de livraison.
Si une demande est urgente, elle peut être délivrée le plus rapidement
possible sans attendre les commandes des ateliers voisins.

Ateliers:
\begin{enumerate}
    \item Commande de pièce de rechange réalisée par un responsable atelier
\end{enumerate}

Magasin:
\begin{enumerate}
    \item Si le magasin ne possède pas les pièces nécessaires
    \begin{enumerate}
        \item Commande des pièces à un fournisseur
    \end{enumerate}

    \item Si la commande est urgente
    \begin{enumerate}
        \itemImpression d'un avis de livraison
        (document utilisé pour confirmé la livraison)
        \item Livraison directe par camion
        Sinon
        \item Attente des commandes de tous les ateliers de la même zone
        \item Impression des avis de livraison
        \item Livraison par zone une fois par semaine
    \end{enumerate}
\end{enumerate}

À chaque fin de travaux dans un chantier ou à chaque nouveau chantier les
«zones chantiers» sont mises à jour. Un calcul est fait à partir des
distances entre les ateliers pour créer des zones où les chantiers sont
distants entre eux en moyenne de 30 km.


\subsection{Gestion des demandes de matériel}

%Dis que l'on garde le même processus, mais que l'on va numériser
%Précise les transferts entre chantiers

Le processus actuel est satisfaisant, et va donc être conservé. Il sera
toutefois numérisé, et formalisera les transferts de matériels entre
chantiers proches, à la fin d'un chantier et si le matériel ne nécessite
pas de maintenance lourde (effectuée au siège).


\subsection{Maintenance}

Le processus actuel est satisfaisant.

\subsection{Planification}

Le diagramme ne précise pas les différents intervenants ; on rappelle que :
\begin{itemize}
\item Chaque chef de chantier réalise avant le début du chantier un premier
planning d'utilisation de matériel;
\item Ce planning est transmis à la Direction du Matériel qui peut tenir à
jour un planning global et si nécessaire lancer des investissements (achat
de gros matériel).
\end{itemize}

\vskip 6pt

De plus, le MCT ne fait pas apparaître un point qui nous semble essentiel :
la correction du planning une fois le chantier lancé. Le chef de chantier
devrait réévaluer son planning d'utilisation de matériel au cours du
chantier, à raison d'une fois par mois. De cette manière, on sera à même de
mieux gérer l'utilisation du matériel et sa maintenance (si un matériel est
libéré plus tôt que prévu par exemple).


\subsection{Facturer les chantiers}

Il processus actuel est satisfaisant, cependant il manque un point essentiel,
la réactivité. \\
Il faudrait pouvoir gérer ou accéder aux informations de facturations en 
temps réel, plutôt que d’attendre l'édition de la facture au début du 
mois suivant.


\end{document}
