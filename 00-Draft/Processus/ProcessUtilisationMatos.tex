% Doit compiler à partir de ce simple fichier
% Mettre NOM et DATE et FICHIER RELATIF
\title{Processus de planification de l'utilisation du matériel - DRAFT}
\author{Cgg}
\date{Vendredi 4 février 2011}

\documentclass[a4paper]{article}

\usepackage[top=2cm, bottom=2cm, left=2cm, right=2cm]{geometry}

\usepackage[utf8x]{inputenc}
\usepackage[french]{babel}

\begin{document}
\maketitle

\section{Terminologie}
Dans ce document, on utilisera les termes et abréviations suivantes :

\begin{description}
\item[CdC] Chef de Chantier
\item[DM] Direction du Matériel
\item[QQOQCP] Qui Quoi Où Quand Comment Pourquoi ; processus
d'exploration d'un sujet
\end{description}

\section{QQOQCP}

\subsection*{Qui}
Ce processus est conduit par le CdC.\\
\textbf{Le processus implique la participation de la DM ;
via quels départements ? (Matériel et Maintenance certainement, Achat ?) ;
de quelle façon ? (dialogue initial puis validation des révisions du
planning ? Dialogue pour les révisions aussi ?)}
Le CdC demande à la DM qui centralise les demandes, effectue un arbitrage
si nécessaire et met en place un planning global de l'affectation du
matériel.

\subsection*{Quoi}
Ce processus consiste à planifier à l'avance les besoins d'un chantier en
terme de matériel. Il est conduit une première fois avant l'ouverture du
chantier (pour déterminer la dotation initiale du chantier en matériel),
puis est régulièrement révisé.

\subsection*{Où}
Lors de la premère itération, le processus est effectué au siège de GSTP ;
par la suite le CdC l'effectue directement sur le chantier.

\subsection*{Quand}
La première version du planning doit être terminée avant le début du
chantier. Les révisions sont faites toutes les deux semaines pour un
chantier standard. Ce délai peut être allongé (chantier stable) ou
raccourci (chantier pas stable). En cas d'imprévu une révision
extraordinaire peut être effectuée.

\subsection*{Comment}
Cf. partie suivante.

\subsection*{Pourquoi}
La mise en place rigoureuse d'un tel processus, allié à l'utilisation d'un
progiciel de gestion globale de la DM, permettra d'atteindre les objectifs
de réduction des coûts et d'amélioration de la qualité.

\section{Description du processus}

Le processus est décomposé de la manière suivante :
\begin{enumerate}
\item Planification initiale pour toute la durée du chantier
\item À intervalles réguliers, ré-évaluation du planning
\item SP3
\item SP4
\item yomama
\end{enumerate}

\end{document}
