Solution Specifique

La solution spécifique répond aux besoins de la Direction Matériel d'informatisation de la gestion de Materiel.
Pour faciliter la gestion de Chantiers, nous proposons d'acheter 30 PCs (portable) pour chaque chantier, 
deux serveurs (l'un pour l'application Chantier, Magasin, etc et l'autre pour la base de données), des cartes 3G
pour chaque chantier pour la connexion au Réseau GSM. 


Le prix d’un PC est autour de : 499 € 30 postes :  14970+ installation 200€ = 16970€
Le prix estimé d’une carte 3G avec le forfait :12€ * 40= 480@
Le forfait 3G est 50€/mois * 40 = 2000€/moins 
Nous avons choisi d’externaliser les serveurs de base de données et d'applocation. Nous avons choisi GoGRID  
comme hébergeur. 
Cela revient donc à un forfait mensuel de 60 euros, soit un forfait annuel de 720euros.


D'autre part la solution spécifique permet d'améliorer considérablement l'efficacité des opérations maintenance,
planification et le déplacement des engins sur les chantiers.
Ce scénario comporte quatre applications à développer, chacune étant  détaillée dans le dossier de la 
solution spécifique.
Pour un développement spécifique, on compte 10 jours/ outils.

Les outils par package d'application :

PACKAGE APPLICATION MATERIEL
1.Analyse de planning du materiel
2.Gestion du Patrimoine
3.Synchronisation/Envoi du planning aux collaborateurs.
4.Récupération des informations et production de factures
5.Envoi des factures et réception des réponses de chantiers

PACKAGE Application ACHAT
1.Ajout  d’un package d’envoi de bons de commandes sur l’application de gestion des 
bons de commande existant.
2.Gestion des Fournisseurs

PACKAGE APPLICATION CHANTIER
1.Outil d’identifiant (login) et accès à la base de données externe
2.Outil de vérification des données (avant l’envoi)
3.Saisie des besoins en matériel du chantier
4.Saisie des heures de travail du personnel
5.Affichage des informations concernant le chantier

PACKAGE APPLICATION MAINTENANCE
1.Recuperation des besoin de maintenance des materiels
2.Etablir Planing pour la maintenance

Au total il y a 14 outils à développer. Une journée de développement coûte environ 300€.
Le prix total du logiciel est de : 200€*14 outils = 4200€

Dans cette solution on propose aussi une formation aux utilisateurs pour les nouvelles applications et outils
développés.On estime qu’on aura besoin d’une journée de formation. Une journée de formation coûte environ 1000€.
Le prix total des 4 packages est de 4000€.
Le temps total de formation est évalué à 10 jours ouvrables

Pour la gestion et maintenance du SI on a besoin d'un seule personne. Elle s’occupera aussi de la formation
du personnel au système et aura un salaire annuel de 24400 euros.

Effectif Direction du Matériel au début : 70 personnes
Département Matériel
- 5 personnes

Département Maintenance
- 3 magasiniers
- 60 personnes

Département Achat
- 2 personnes

Effectif Direction du Matériel après l'informatisation : 5 personnes
Département Matériel
- 1 personne pour la gestion du matériel
- 1 personne pour la facturation du matériel
Département Achat
- 1 personne pour la location et restitution du matériel
- 1 personne pour l’achat de matériel non louable
Service SI
o 1 personne pour la gestion et maintenance du SI : 2200 euros/mois

Concernant la maintenance informatique, un service de maintenance du SI est implanté, 
avec une personne qui s’occupe de la gestion du SI.
