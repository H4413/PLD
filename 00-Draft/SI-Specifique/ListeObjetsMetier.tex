% Doit compiler à partir de ce simple fichier
% Mettre NOM et DATE et FICHIER RELATIF


% Insérez ici le titre de votre draft, votre nom et la date de rédaction.

\documentclass[a4paper]{article}

\usepackage[utf8x]{inputenc}

\title{PLD - SI Spécifique\\Liste des objets métier à développer}
\author{Cgg\\Karen}
\date{18 février 2011}

\usepackage[top=2cm, bottom=2cm, left=2cm, right=2cm]{geometry}

\usepackage[french]{babel}

\usepackage{slashbox}

\begin{document}
\maketitle

\begin{abstract}
Liste des objets métier à développer pour le SI spécifique de GSTP. La
plupart des objets sont repris du MCD existant, et on en ajoute quelques
un (liés à la planification).
\end{abstract}

\section{Liste des objets métiers à développer}

La liste des objets métiers est présentée sous la forme d'un tableau à deux
entrées. De cette façon, on peut plus facilement voir quels sont les objets
métiers réutilisés plusieurs fois par différentes application.\\

\begin{center}
\begin{tabular}{|c|c|c|c|c|}
\hline
\backslashbox{Listes Objets Métiers}{Applications}&Achats& Matériel&Maintenance&Chantier\\
\hline
Affectation&&x&&x\\
\hline
Atelier&&x&x&\\
\hline
Chantier&&x&x&\\
\hline
Commande&x&x&x&x\\
\hline
Facture&&x&&x\\
\hline
Fonction&&&x&\\
\hline
Catalogue fournisseurs&x&&&\\
\hline
Fournisseur&x&&&\\
\hline
Gamme Maintenance&&&x&\\
\hline
Livraison&x&&&\\
\hline
Matériel&&x&x&x\\
\hline
Opération Maintenance&&&x&x\\
\hline
Période&&x&x&x\\
\hline
Piece Rechange&x&&x&x\\
\hline
Planification d'affectation&&x&x&x\\
\hline
Planification de maintenance&&x&x&\\
\hline
Poste Fonctionnel&x&x&x&x\\
\hline
Prestation&x&&&\\
\hline
Produit Acheté&x&&&\\
\hline
Regroupement de produits&&x&&\\
\hline
Type de matériel&&x&&\\
\hline
Type d'opération de maintenance&&&x&\\
\hline
Type de période&x&x&x&x\\
\hline
Type de prestation&x&&&\\
\hline
Type de regroupement de produit&&x&&\\
\hline
\end{tabular}
\end{center}

\end{document}
