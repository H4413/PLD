%!TEX encoding = UTF-8 Unicode
\documentclass [a4paper] {report}
\usepackage[utf8]{inputenc}
\usepackage[francais]{babel}
\usepackage[top=2cm, bottom=2cm, left=2cm, right=2cm]{geometry}
\usepackage{hyperref}
\usepackage{eurosym}

\begin {document}

\section{Applications}

On présente ici la liste des fonctions de chaque application composant le
SI spécifique conçu pour GSTP.


\subsection{Application Achats}
% Suivre commande
% Communication avec la comptabilité pour les commandes
% Suivre le budget

\begin{itemize}
\item Commander du matériel
\item Commander des pièces de rechanger
\item Consulter le catalogue de fournisseurs (recherche de fournisseurs selon 
un ou plusieurs critères)
\item Editer le catalogue de fournisseurs :
	\begin{itemize}
	\item Création d'une fiche fournisseur
	\item Modification d'une fiche fournisseur
	\item Suppression d'une fiche fournisseur
	\end{itemize}
\end{itemize}

\subsection{Application Matériel}
% Valorisation du Parc
\begin{itemize}
\item Récupération des plannings prévisionnels d'utilisation de matériels des chantiers
\item Etablir le planning effectif d'affectation des matériels pour les chantiers
\item Calculer et transmettre les factures qui correspondent à chaque chantier
à partir de l'utilisation des matériels
\item Enregistrer les mouvements de matériels (entrées et sorties du parc)
\end{itemize}

\subsection{Application Maintenance}
% Saisie des temps des ouvriers qui font la maintenance
% Gestion du catalogue de maintenance (changer une roue il faut 3h...)
% Planification de l'affectation des ouvriers à la maintenance
% Etablir le planning journalier
% Anticiper les planifications curatives
\begin{itemize}
\item Récupération de l'ensemble des demandes de maintenance exceptionnelle des matériels
\item Etablir le planning pour la maintenance du matériel
\item Editer le journal de maintenance pour chaque matériel :
    \begin{itemize}
    \item Modification d'une fiche de maintenance % A préciser
    \item Pas de création/suppression car la fiche de maintenance est liée
    au matériel auquel elle se réfère. La fiche de maintenance est donc
    créée puis supprimée automatiquement en même temps que le matériel dans
    la base.
    \end{itemize}

%Gestion du magasin
% Ensemble des fonctionnalités pour gérer un magasin
% Gestion des entrées/sorties (des PR...) (Gestion des livraisons)
% Gestion des Stocks
% Gestion des Commandes de PR et des livraisons
% Groupement des commandes des chantiers
% Outils pour suivre la livraison

\end{itemize}

\subsection{Application Chantier}
% Suivre son budget
% Application Chantier pourra être plus global, c'est les fonctions adaptées aux
% départements maintenance
% Saisie mensuel du temps prévue d'utilisation
% Saisie hebdomadaire du temps
% Commande des PR
\begin{itemize}
\item Prévision d'utilisation réelle du matériel
\item Saisie des heures de travail du personnel
\item Affichage de prochaines livraisons
\item Affichage l'état des factures en cours
\end{itemize}

%Toutes les applications disposeront des fonctionnalités suivantes : 
% - Authentification
% ...

\end{document}
