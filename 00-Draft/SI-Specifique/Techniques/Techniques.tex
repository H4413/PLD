\documentclass[a4paper]{article}

\usepackage[utf8x]{inputenc}
\usepackage[french]{babel}

\title{Annotation du MCT - DRAFT}
\author{Cgg, Victor}
\date{6 février}

\usepackage[top=2cm, bottom=2cm, left=2cm, right=2cm]{geometry}

\begin{document}

\section{Architecture Technique}
L'amélioration du système d’information de l'entreprise passe tout d'abord par une augmentation du nombre de postes
informatiques. Toutes les informations passeront par le système informatique et seront enregistrées.

\subsection{Schéma}

 %\begin{figure}[!h]
  %  \begin{center}
   % \includegraphics[width=4cm]{\PIXPATH/Schema}
    %\caption{}
    %\end{center}
    %\end{figure}


\subsection{Serveur}

Le serveur sera externalisé et il servira de support pour la base de données et la gestion des accès à ces données. 

Chaque application, et donc chaque poste, sera configurée pour se connecter à la base de données du serveur. 
Toutes partageront les mêmes informations et il sera ainsi possible de facilement faire le rapprochement entre 
les achats et le matériel réceptionné par exemple.

	\subsubsection{Architecture}
Nous avons besoin d'un serveur d'application qui hébergera les applications pour le chantier, le magasin, et les différents 
départements,  ainsi qu'une base de données pour les information concernant aux chantiers, magasin, fournisseurs, etc. 

Pour établir la communication client-serveur on utilisera le protocole LDAP. Il fournit à l'utilisateur des commandes 
pour se connecter ou se déconnecter,  pour rechercher, comparer, créer, modifier ou effacer des entrées. 
Des mécanismes de chiffrement (SSL ou TLS) et d'authentification (SASL), couplés à des mécanismes de règles d'accès (ACL) 
permettent de protéger les transactions et l'accès aux données.

	\subsubsection{Hébergement}
    Il nous faut un ensemble de serveurs capables de répondre à un grand
    nombre de requêtes, et ce en permanence.

Une solution consiste à héberger nos serveurs dans le \textsl{cloud}, on utilisera le modèle IAAS pour  que l'entreprise 
maintient ses applications, les {\sl runtimes}, l'intégration SOA, les bases de données, le logiciel serveur.
Cette solution nous permettant ainsi de payer uniquement les ressources que nous consommons, tout en bénéficiant 
d'une bonne résilience aux pannes et d'une bonne disponibilité.
L'entreprise GoGrid fournit un tel service, à partir de 60 \euro par mois.


\subsection{Communication}
Le département Achat, Matériel et Maintenance doivent être en réseau locale pour qu'ils puissent  communiquer entre elles.

Sur les chantiers, il faut pouvoir relier les ordinateurs au siège par l'intermédiaire d'Internet. 
Nous savons que les chantiers
 sont souvent amenés à se déplacer dans le cadre de leur travail, donc le réseau 3G est indispensable 
pour qu'ils puissent être plus réactif vis-à-vis de leur activité, comme par exemple, répondre 
instantanément à un email, rechercher une information, ou échanger et communiquer sur une messagerie. 

Chaque poste disposera d’une clé 3G pour se connecter au réseau GSM haut débit.

Il existe une large choix de formules proposées par les différents opérateurs de téléphones portables, 
pour citer quelques unes:

Forfait journée: connexion n'importe quand durant tout la journée
Prix: 2 à 12 \euro

Forfait mois illimités: navigation par nombre d'heures illimitées.
Prix: de 30 \euro à 70 \euro


   

\subsection{Chantiers}
On ajoutera 30 PC avec des caractéristiques qui correspondent au milieu de travail (des ordinateurs résistantes 
au pression, vibration, humidité, etc. 1 pour chaque chantier), ces ordinateurs n'ont pas besoin d’être très puissants, 
seulement ils doivent se connecter à internet et être capable de faire fonctionner l'application de chantier.

Exemple:
DELL ATG D620: PC Portable qui résiste l'humidité, l'altitude, pression, et vibration, protégé 
contre les coups.
	
	Caractéristiques:

	\begin{enumerate}
		\item 14,1 puces
		\item lumineux
		\item sans reflet
		\item RAM 4 Go, Core 2 Duo T7600
		\item 80 Go, 4200 T/M Anti-choc
		\item Windows Vista
		\item 2,8 Kg
	\end{enumerate}

<<<<<<< HEAD
Prix: 499 euros
=======
Prix: 499 \euro
>>>>>>> 4d73a8319c46aeaa83eba980769ed70fed51b57d




\end{document}
