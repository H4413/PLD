% Doit compiler à partir de ce simple fichier
% Mettre NOM et DATE et FICHIER RELATIF

\documentclass[a4paper]{article}

\usepackage[top=2cm, bottom=2cm, left=2cm, right=2cm]{geometry}

\usepackage[utf8]{inputenc}
\usepackage[french]{babel}

\title{Draft : Etude de l'existant\\Département Achat}
\author{Clément}
\date{14 janvier 2010}

\begin{document}
\maketitle

\section{Département achat}


\subsection{Processus majeurs}

Le département d'achat s'occupe principalement de l'achat de matériel et de
pièces de rechange ; il réalise également des achats de prestations en cas
de besoin et gère son catalogue de fournisseurs.

Achat matériel différent achat PR : prix, acte d'investissement autorisé
par direction ou fonctionnement normal, délai


\subsubsection{Achat de matériel}

La direction du matériel déclenche une demande d'achat de matériel auprès
de la direction générale lorsqu'elle constate une pénurie de matériel. La
DG effectue si nécesaire un arbitrage entre les différentes demandes.\\
Lorsqu'une demande est acceptée, elle est confiée au département achat et
la procédure ci-après est suivie :

\begin{enumerate}
\item Choix d'un fournisseur
\item Négociation
\item Formalisation d'un contrat d'achat
\item Travail de suivi de la commande jusqu'à réception
\item Réception de commande, contrôle
\item Si problème ou réserve : retour de la commande au fournisseur ; sinon
le matériel est intégré au parc
\end{enumerate}


\subsubsection{Achat de pièces de rechanges}

Les achats des diverses pièces de rechange sont déclenchés par le magasinier.
Ces achats sont différents des achats de matériel : ils font partie du
fonctionnement normal de l'entreprise, les délais sont très courts et les
prix ne sont pas comparables à ceux d'achat de matériel.\\
Dans les cas normaux, les achats de pièce de rechange sont donc
quasi-automatique ; toutefois, il arrive parfois que le département achat
suive la procédure appliquée dans le cas d'un achat de matériel, lorsqu'il
faut commander un nouveau type de pièces de rechanges par exemple.


\subsubsection{Achat de prestations}

Les prestations sont principalement des locations de matériel auprès d'entreprises extérieures 
en cas de matériel indisponible au sein de la société GSTP, quelque soit la
raison.\\
La location est une solution de réactivité, transparente pour le chantier.

augmenter la
précision des intervenants si connus

\begin{enumerate}
\item Commande 
\item Suivi de commande
\item Soldage de commande
\end{enumerate}


\subsubsection{Gestion des fournisseurs}

Le département achat doit maintenir à jour son catalogue de fournisseur.
Cela inclut les étapes suivantes :
\begin{enumerate}
\item Recherche de fournisseur
\item Prise de contact
\item Maintien à jour du catalogue de fournisseur (ajout, suppression et
modification de fournisseurs)
\end{enumerate}


\subsection{Objets métiers}

La liste des objets métiers identifiée est la suivante :

\begin{itemize}
\item Direction du Matériel
\item Direction Générale
\item Département Achat
\item Magasinier
\item Fournisseur
\item Commande
\item Livraison
\item Matériel
\item Pièce de rechange
\item Prestation
\end{itemize}


\subsection{SI}

\subsection{Matériel}
Le département Matériel dispose de deux PC et deux imprimante (un couple
PC/Imprimante par employé du département).

\subsection{Logiciel}
Le département achat utilise les applications suivantes :
\begin{itemize}
\item Gestion des fournisseurs (environ 300 fournisseurs)
\item Gestion des bons de commande (2 à 3 achats de gros matériel par mois
et 600 achats de petits matériels par an).
\end{itemize}

Les applications ont été développées en interne.


\subsection{Conclusions : +/-}

Les procédures employées par le département Achat sont bien rodées et
fonctionnent de manière satisfaisante. Toutefois, l'infrastructure
informatique laisse à désirer ; les logiciels développés en interne sont
sans doute peu évolutifs et peu ou pas interfacés avec ceux utilisés par la
Direction du Matériel ou la Direction Générale, par exemple.

\end{document}
