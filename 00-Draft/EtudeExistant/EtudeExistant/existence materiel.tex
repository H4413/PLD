%!TEX encoding = UTF-8 Unicode
\documentclass [a4paper] {report}
\usepackage[utf8]{inputenc}
\usepackage[francais]{babel}
\usepackage[top=2cm, bottom=2cm, left=2cm, right=2cm]{geometry}
\usepackage{hyperref}
\usepackage{eurosym}
\begin {document}

\section{Département Matériel}
3 services : \\
\begin{itemize}

\item	Service Gestion du Matériel\\
	\begin{itemize}
		\item	planification/affectation du matériel aux chantiers\\
	\end{itemize}
\item	Gestion du Parc Matériel \\
	\begin{itemize}
		\item réception/envoi du matériel\\
		\item gestion du parc\\
	\end{itemize}
\item	Facturation Matériel \\
	\begin{itemize}
		\item valorisation/facturation du matériel\\
	\end{itemize}		

\end{itemize}		
		
\hfill\\
	
2 Procédures: Procédure Facturation et Procédure Affectation\\\\

\subsection{Procédure Facturation}
	Le processus de Facturation est réalisé par le service Facturation Matériel. La facture(mensuel) de matériel est calculée a partir de 
	la valorisation du matériel plus le cout de maintenance:\\
\begin {enumerate}
	\item La valorisation du Matériel
		Calculé a partir de \\
		\begin{itemize}
			\item Pointage d'utilisation du Materiel\\
			\item Avis valorisation Frais de Structure génère par la gestion des matériels	\\
		\end{itemize}	
	\item Coût de maintenance
		Le calcul du cout de la maintenance est déterminée à partir de:\\
		\begin{itemize}
			\item Avis Valorisation Personnes: c'est le Coût de la main d'ouvre  consomme par le matériel\\
			\item Avis Valorisation Pieces de Rechange et pneus: Dépense en piece de rechange et pneus \\
			\end{itemize}
		\end{itemize}
\end{enumerate}
\hfill\\

Liste des objets métiers :
\begin{itemize}
	\item	Facture\\
	\item Chantier\\
	\item Fournisseur\\
	\item livraison
	\item Matériel\\
	\item Piece Rechange\\
\end{itemize}

Dysfonctionnement:

source : GSTP/Ressources/Modele-de-l-existant/MCT-Facturer-chantier.doc

\hfill\\
\hfill\\
\subsection{Procédure Affectation du matériel}

	Pour planifier l'affectation du matériel on distingue :\\
	\begin{itemize}
		\item La affectation d'utilisation de matériel, est en fonction des demandes des chantier pour le matériel.\\
		\item La demande de location de matériel, si aucun matériel est disponible pour satisfaire une demande. C'est le service de location qui fait la gestion de location.\\
		\item L'autorisation d'approvisionnement, validée à partir d'un calcul de besoin en fonctions de variation des stocks.\\
		\item La maintenance préventive du matériel effectuée après un certain temps d'utilisation.\\
	\end{itemize}
\hfill\\
Liste des Objets métiers:\\
\begin{itemize}
	\item	Affectation\\
	\item Matériel\\
\end{itemize}


source : GSTP/Ressources/Modele-de-l-existant/MCT-Planification.doc



\hfil\\

\subsection{SI}\\
\begin{enumerate}

\item  \textbf{Materiel}\\

Le departement de Materiel dispose de:\\

\begin{itemize}
\item 3 pcs et 2 Imprimantes\\
\end{itemize}

\item  \textbf{ Logiciel}\\

Le département Matériel utilise les applications suivants:\\

\begin{itemize}
\item  Gestion du Planning(Matériels)\\
\item  Facturation(Matériels)
\end{itemize}
\end{enumerate}
\end{document}


