\documentclass[a4paper]{article}

\usepackage[top=2cm, bottom=2cm, left=2cm, right=2cm]{geometry}

\usepackage[utf8]{inputenc}
\usepackage[french]{babel}

\title{Draft : Etude existant, conclusion}
\author{Clément}
\date{20 janvier 2010}

\begin{document}
\maketitle


\section{Synthèse de l'étude de l'étude de l'existant}

De manière globale, on peut constater un manque de souplesse/d'agilité dû
aux points suivants :


\subsection{Equipement matériel disparate}
Tous les départements sont équipés mais tous les chantiers ne le sont pas ;
cela induit des retards dus à la transmission de l'information, aux erreurs
de saisies dans le système, etc.  

\subsection{Equipement logiciel disparate}
Les différents départements utilisent différentes applications de gestion
développées en interne. Cela conduit à des logiciels peu ou pas documentés
et évolutifs. Les formats de données utilisés ne sont pas standards,
rendant difficile les interconnections de logiciels de différents
départements. De manière générale, le SI n'est pas urbanisé : pas ouvert,
pas communiquant, pas évolutif.

\subsection{Pas de notion d'urgence clairement définie}
Les procédures définies fonctionnent mais aucune notion de hiérarchisation
n'apparaît nulle part concernant les demandes (affectation, achat, de
matériels ou pièces de rechange).  Il ne semble pas possible de qualifier
une demande de matériel ou de maintenance comme étant urgente ou critique.

\subsection{Pas de décision}

De manière général, l'entreprise ne possède pas d'outil pour permettre la 
décision. C'est à dire de tableaux de bords et de reporting.

\subsection{Conclusion}
Au terme de l'étude de l'existant dans l'entreprise, on dispose d'un audit
portant sur le domaine de l'entreprise impacté le projet. Cet audit nous
permet de mieux cerner l'organisation du domaine et la manière dont il
fonctionne, afin de mieux réaliser le benchmark dans la partie suivante du
dossier. Il permet également de pointer les faiblesses de l'existant
surlesquelles nous seront amener à concentrer nos efforts.

\end{document}
