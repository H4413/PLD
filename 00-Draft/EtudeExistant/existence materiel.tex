%!TEX encoding = UTF-8 Unicode
\documentclass [a4paper] {report}
\usepackage[utf8]{inputenc}
\usepackage[francais]{babel}
\usepackage[top=2cm, bottom=2cm, left=2cm, right=2cm]{geometry}
\usepackage{hyperref}
\usepackage{eurosym}
\begin {document}

\section{Département Matériel}
\subsection{Organisation des services du Dpt Matériel}


	\subsubsection{Gestion du Matériel}

	Cet service composée de 3 personnes, se charge de planifier l'affectation du matériel ainsi que les différents zones de maintenance.\\
	
	\begin{itemize}
		\item	planification/affectation du matériel aux chantiers\\
	\end{itemize}
	
	\subsubsection{Gestion du Parc Matériel}
	Service composée d'une personnes s'occupe de la réception du matériel(Demande d'entré de matériel au parc matériel central), de l'envoi du matériel(Demande de sortie de matériel du parc matériel central), de la gestion d'entrées de matériel( Entrée effective de matériel au garage après acceptation) et de la sortie de Matériel(Sortie effective de matériel du garage après acceptation).\\
	
	\begin{itemize}
		\item réception/envoi du matériel\\
		\item gestion du parc\\
	\end{itemize}
	
	\subsubsection{Facturation Matériel}
	Cet service composée d'une personne a comme principal fonction de facturer correctement un chantier selon une valorisation du matériel et un calcul de maintenance du matériel(cout de maintenance, cout de pièce de rechange, cout de personnel)\\
	
	\begin{itemize}
		\item valorisation/facturation du matériel\\
	\end{itemize}		

\subsection{Procédures}
	
Le département Matériel est composée de 2 Procédures principales: Procédure Facturation et Procédure Affectation

\subsubsection{Procédure Facturation}
	Le processus de Facturation est réalisé par le service Facturation Matériel. La facture(mensuel) de matériel est calculée a partir de 
	la valorisation du matériel plus le cout de maintenance:\\
\begin {enumerate}
	\item La valorisation du Matériel
		Calculé a partir de \\
		\begin{itemize}
			\item Pointage d'utilisation du Materiel\\
			\item Avis valorisation Frais de Structure génère par la gestion des matériels	\\
		\end{itemize}	
	\item Coût de maintenance
		Le calcul du cout de la maintenance est déterminée à partir de:\\
		\begin{itemize}
			\item Avis Valorisation Personnes: c'est le Coût de la main d'ouvre  consomme par le matériel\\
			\item Avis Valorisation Pieces de Rechange et pneus: Dépense en piece de rechange et pneus \\
		\end{itemize}

\end{enumerate}

source : GSTP/Ressources/Modele-de-l-existant/MCT-Facturer-chantier.doc

\hfill\\
\hfill\\
\subsection{Procédure Affectation du matériel}

	Pour planifier l'affectation du matériel on distingue :\\
	\begin{itemize}
		\item La affectation d'utilisation de matériel, est en fonction des demandes des chantier pour le matériel.\\
		\item La demande de location de matériel, si aucun matériel est disponible pour satisfaire une demande. C'est le service de location qui fait la gestion de location.\\
		\item L'autorisation d'approvisionnement, validée à partir d'un calcul de besoin en fonctions de variation des stocks.\\
		\item La maintenance préventive du matériel effectuée après un certain temps d'utilisation.\\
	\end{itemize}
\hfill\\

source : GSTP/Ressources/Modele-de-l-existant/MCT-Planification.doc


\subsection{Liste des objets métiers}
\begin{itemize}
	\item	Facture\\
	\item Chantier\\
	\item Fournisseur\\
	\item livraison\\
	\item Affectation\\
	\item Matériel\\
	\item Piece Rechange\\
\end{itemize}


\subsection{Système d'Information}
\begin{enumerate}

\item  \textbf{Materiel}\\

Le departement de Materiel dispose de 3 pcs( pour lancer l'application de facturation et de planning), qui sont relies au serveur de l'entreprise et 2 Imprimantes nécessaires pour imprimer les factures de chaque chantier\\

\item  \textbf{ Logiciel}\\

Le département Matériel utilise une application de facturation, et l'application de gestion de planning( pour les materiel)\\
\end{enumerate}

\subsection{Dysfonctionnements}
Les dysfonctionnements sont contre productifs, enclenchant ainsi une dynamique négative qui fragilise l'organisation. L'objectif est d'analyser les possibles dysfonctionnements dans les différents départements et d'envisager 
possibles solutions pour renforcer l'organisation.

\subsubsection{Département Matériel}

	
	
\end{document}




