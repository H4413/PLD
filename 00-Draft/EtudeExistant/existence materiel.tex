%!TEX encoding = UTF-8 Unicode
\documentclass [a4paper] {report}
\usepackage[utf8]{inputenc}
\usepackage[francais]{babel}
\usepackage[top=2cm, bottom=2cm, left=2cm, right=2cm]{geometry}
\usepackage{hyperref}
\usepackage{eurosym}
\begin {document}

\section{Département Matériel}
\subsection{Organisation des services du Dpt Matériel}


	\subsubsection{Gestion du Matériel}

	Ce service est composée de 3 personnes. Il se charge de planifier l'affectation du matériel ainsi que les différentes zones de maintenance.\\
	
	\subsubsection{Gestion du Parc Matériel}
	
	Ce service est composée d'une personne qui s'occupe de la réception du matériel (Demande d'entrée de matériel au parc matériel central), de l'envoi du matériel (Demande de sortie de matériel du parc matériel central), de la gestion d'entrée de matériel(Entrée effective de matériel au garage après acceptation) et de la sortie de Matériel(Sortie effective de matériel du garage après acceptation).
		
	\subsubsection{Facturation Matériel}
	Ce service est composée d'une personne qui a comme fonction principale de facturer correctement un chantier en fonction de la valorisation du matériel et du calcul du coût de la maintenance du matériel (coût de maintenance, coût de pièce de rechange, coût de personnel)\\
	
\subsection{Procédures}
	
Le département Matériel a 2 Procédures principales, les procédures de facturation et les procédures d'affectation.

\subsubsection{Procédure Facturation}
	Le processus de Facturation est réalisé par le service Facturation Matériel. La facture (périodicité mensuelle) de matériel est calculée a partir de 
	la valorisation du matériel et du cout de maintenance:\\
\begin {enumerate}
	\item La valorisation du Matériel
		Calculé a partir de \\
		\begin{itemize}
			\item Pointage d'utilisation du Materiel\\
			\item Avis valorisation Frais de Structure génère par la gestion des matériels	\\
		\end{itemize}	
	\item Coût de maintenance
		Le calcul du cout de la maintenance est déterminée à partir de:\\
		\begin{itemize}
			\item Avis Valorisation Personnes: c'est le Coût de la main d'ouvre  consomme par le matériel\\
			\item Avis Valorisation Pieces de Rechange et pneus: Dépense en piece de rechange et pneus \\
		\end{itemize}

\end{enumerate}

source : GSTP/Ressources/Modele-de-l-existant/MCT-Facturer-chantier.doc

\hfill\\
\hfill\\
\subsection{Procédure Affectation du matériel}

	Pour planifier l'affectation du matériel on distingue :\\
	\begin{itemize}
		\item L'affectation d'utilisation de matériel qui est faite en fonction des demandes des chantier pour le matériel.\\
		\item La demande de location de matériel, faites si aucun matériel est disponible pour satisfaire une demande. C'est le service de location qui fait la gestion de location.\\
		\item L'autorisation d'approvisionnement, validée à partir d'un calcul de besoin en fonctions de variation des stocks.\\
		\item La maintenance préventive du matériel efc



\end{document}




