% Doit compiler à partir de ce simple fichier
% Mettre NOM et DATE et FICHIER RELATIF
\title{Analyse des besoins de GSTP - DRAFT}
\author{Cgg}
\date{28 janvier 2011}

\documentclass[a4paper]{article}

\usepackage[top=2cm, bottom=2cm, left=2cm, right=2cm]{geometry}

\usepackage[utf8]{inputenc}
\usepackage[french]{babel}

\begin{document}
\maketitle

\hfill\\

\section{Analyse des besoins de GSTP}

Après avoir analysé la manière dont fonctionne la direction du matériel de
GSTP dans la première partie de ce dossier, on a pu faire ressortir
plusieurs dysfonctionnements, matériels et logiciels. Les
dysfonctionnements matériels peuvent être pris en compte directement par
la société et ne sont pas l'objet de ce projet. Concernant les
dysfonctionnements logiciels, on peut en tirer une liste des besoins de la
Direction du Matériel, qui nous permettra d'orienter notre choix de l'ERP
qui sera le coeur du nouveau SI de la Direction du Matériel.\\

Besoins de la Direction du Matériel en matière d'applicatif :

\begin{description}
\item[SI communiquant] L'information ne doit pas être cloisonnée entre
différents modules
\item[SI transversal] Le même logiciel doit être utilisé dans tous les
départements de la direction du matériel
\item[SI évolutif] Si la société GSTP décide de modifier l'architecture de
sa direction du matériel ou d'étendre le renouvellement du SI à d'autres
départements, cela doit être possible à partir de l'existant sans grosses
contraintes
\item[Outils] L'ERP retenu doit intégrer les outils suivants :
    \begin{itemize}
    \item reporting
    \item planification
    \item aide à la décision
    \item business intelligence (oh oui !)
    \end{itemize}
\end{description}

\end{document}
