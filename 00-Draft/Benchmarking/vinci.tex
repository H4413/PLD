% Doit compiler à partir de ce simple fichier
% Mettre NOM et DATE et FICHIER RELATIF
\title{Benchmark Vinci - DRAFT}
\author{VB}
\date{28 janvier 2011}

\documentclass[a4paper]{article}

\usepackage[top=2cm, bottom=2cm, left=2cm, right=2cm]{geometry}

\usepackage[utf8]{inputenc}
\usepackage[french]{babel}

\begin{document}
\maketitle

\hfill\\
\section {VINCI Construction}

\subsection{Présentation du groupe}

Avec un chiffre d’affaires de 6,7 milliards d’Euros en 2008, Vinci Construction France est le leader du BTP en France suite au rapprochement des 2 réseaux de BTP français appartenant au Groupe Vinci : SOGEA et GTM \\

Il réunit un ensemble sans équivalent de compétences dans les métiers du bâtiment, du génie civil, des travaux hydrauliques et des services.
Son activité se répartit en trois composantes complémentaires : \\

\begin{itemize}
\item la France métropolitaine, avec Vinci Construction France, formée du rapprochement en 2007 de Sogea Construction et de GTM Construction, qui dispose d’un réseau de 370 centres de profit fortement ancrés sur leurs marchés régionaux et locaux ;
\item les marchés locaux hors France métropolitaine, couverts par un réseau de filiales qui exercent l’ensemble des métiers de la construction dans leurs zones d’intervention : VINCI PLC au Royaume-Uni ; CFE (détenu à 46,8 \%) au Benelux ; Vinci Construction Filiales Internationales en Allemagne, en Europe centrale, dans la France d’outre-mer et en Afrique ;
\item les activités relevant de marchés mondiaux : les grands ouvrages, avec Vinci Construction Grands Projets ; le génie civil spécialisé à haute technicité, avec Freyssinet ; le dragage, avec DEME (détenu à 50 \% par CFE).
\item Vinci Construction est la matrice de la culture d’entrepreneur du Groupe et de son schéma de management, conjuguant travail en réseau, organisation décentralisée, autonomie et responsabilité individuelle de l’encadrement.
\end{itemzine}

Associé à une stratégie constante de développement de la valeur ajoutée et de la part récurrente de l’activité, ce schéma a permis d’enregistrer une progression constante des résultats au cours des cinq dernières années, dans un contexte de forte croissance de l’activité.

\subsection{ERP}

Vinci Construction France utilise Sage X3. D'après VINCI Contruction, Sage propose une solution pérenne, traitant les particularités du métier du BTP telles que la facturation à l’avancement ou la gestion des acomptes. En outre, il répond à une organisation décentralisée grâce à sa technologie SAFE X3. Son interface moderne et son ergonomie conviviale en font un outil facile à appréhender. Et surtout, Sage X3 Finances permet, grâce à sa puissance et à sa richesse fonctionnelle, d’être mieux armé pour produire les analyses et informations financières à destination du groupe. \\

Le groupe a choisi ce ERP, car, d'après eux, le logiciel est capable de répondre la plus part de ces besoins, comme: la dématérialisation des documents, le portail utilisateur, la Business Intelligence et l’accès web.

\end{document}
