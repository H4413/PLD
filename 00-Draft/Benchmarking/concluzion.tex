% Doit compiler à partir de ce simple fichier
% Mettre NOM et DATE et FICHIER RELATIF
\title{Conclusion benchmarking - DRAFT}
\author{Cgg}
\date{28 janvier 2011}

\documentclass[a4paper]{article}

\usepackage[top=2cm, bottom=2cm, left=2cm, right=2cm]{geometry}

\usepackage[utf8]{inputenc}
\usepackage[french]{babel}


\begin{document}
\maketitle

\begin{quote}
This ends today.\\
\em Rambo
\end{quote}

\subsection{Conclusion}

Nous avons examiné les pratiques de deux entreprises leader dans le domaine
du BTP : Vinci et Bouygue. Pour chacune, nous avons également relevé les
ERP utilisés :

\begin{itemize}
\item Vinci utilise Sage X3
\item Bouygue utilise SAP
\end{itemize}

\vskip 6pt

Par ailleurs, nous avons évalué en dehors du contexte d'entreprise les possibilité des ERP suivants :

\begin{itemize}
\item SAP All-in-One
\item BRZ Pharos
\item PROGIB
\item Winlog
\item SAGE X3
\end{itemize}

\vskip 6pt

Afin d'effectuer un choix d'ERP, il nous faut à présent cerner plus
précisément les besoins de la Direction du Matériel de la société GSTP.
\end{document}
