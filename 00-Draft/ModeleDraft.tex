% Doit compiler à partir de ce simple fichier
% Mettre NOM et DATE et FICHIER RELATIF

\usepackage[utf8x]{inputenc}

% Insérez ici le titre de votre draft, votre nom et la date de rédaction.
\title{}
\author{}
\date{}

\documentclass[a4paper]{article}

\usepackage[top=2cm, bottom=2cm, left=2cm, right=2cm]{geometry}

\usepackage[french]{babel}

\begin{document}
\maketitle

\begin{abstract}
% Dans cette section, donnez des instructions concernant l'utilisation du
% draft : livrable dans lequel il s'inclut, état d'avancement (pas fini,
% presque fini, fini, fini-on-n-y-touche-plus), et toute autre info que
% vous jugeriez utile sur le moment (inutile d'inclure la température
% extérieure toutefois).
\end{abstract}

\section{Introduction}

\section{}

\subsection{}

\end{document}
