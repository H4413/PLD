% Doit compiler à partir de ce simple fichier
% Mettre NOM et DATE et FICHIER RELATIF

\usepackage[utf8x]{inputenc}

% Insérez ici le titre de votre draft, votre nom et la date de rédaction.
\title{Commentaire sur le Rapport ARIS}
\author{H4413}
\date{01/01/2011}

\documentclass[a4paper]{article}

\usepackage[top=2cm, bottom=2cm, left=2cm, right=2cm]{geometry}

\usepackage[french]{babel}

\begin{document}
\maketitle

\section{Introduction}

Ce document relate les différentes remarques relatives au rapport ARIS généré. 

\section{Couverture}

La couverture du département Matériel de l'entreprise GSTP n'est pas complet 
dans la modélisation. Les processus de planification ne sont par expemple pas 
modélisés. Le service achat est inclus dans les chaînes de plus-value des autres 
services.

\section{SAP}

Nous nous sommes limités à la représentation de 2 processus SAP sur ARIS pour 
des raisons de temps.

Cependant cela est suffisant pour mettre en exergue la correspondance entre les 
processus de GSTP et de SAP.

\section{Matrices}

Une matrice reliant les postes de travail et les fonctions a été créée à titre 
indicatif.

La deuxième matrice relie les fonctions et les transactions SAP aux fonctions 
de GSTP.

On affiche 2 niveaux de granularité.

Le premier est la correspondance direct entre un scénario SAP et les besoins 
d'un département. 

Le deuxième niveau est le détail entre les fonctions et transacations SAP avec 
les fonctions de GSTP.

On note ainsi les trous fonctionnels entre SAP et nos besoins pour GSTP.

\end{document}
