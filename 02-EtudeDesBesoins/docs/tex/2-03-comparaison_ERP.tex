\pagebreak
\subsection{Comparaison des ERP}

\hskip -15mm
\begin{longtable}{|m{2cm}|m{25mm}|m{25mm}|m{25mm}|m{25mm}|m{25mm}|}
\hline
Critères&SAP All-in-One&BRZ Pharos&PROGIB&Winlog&SAGE X3\endhead
\hline
Type&Global&Spécifique&Spécifique&Spécifique&Global\\
\hline
Gestion des devis&OUI&OUI&OUI&&OUI\\
\hline
Planification &OUI&OUI&OUI&OUI&OUI\\
\hline
Suivi du chantier&&OUI&partiel&OUI&OUI\\
\hline
Gestion de la main d’œuvre&&OUI&OUI&&\\
\hline
Gestion de l’achat et des stocks&OUI&OUI&OUI&OUI&OUI\\
\hline
Facturation et avancement&OUI&OUI&OUI&OUI&OUI\\
\hline
Comptabilité&OUI&OUI&OUI&OUI&OUI\\
\hline
Gestion du matériel&OUI&OUI&OUI&OUI&\\
\hline
Coût&3000 euros/poste&NC&NC&NC&1700 euros/licence\\
\hline
utilisé par&Bouygues, Thalès, EDF&PME (SEGEX, QUINTANA, ...)
&TFN Multiservices (28000 salariés)&PME&Vinci, Eiffage\\
\hline
possibilité d'évolution&très avancées et inclues dans l'offre All-in-One
&Spécifique au BTP, mais convient à tous les besoins de l'entreprise
&Achat possible de modules supplémentaires.
Spécifique aux métiers du bâtiment.
&Entièrement personnalisable&Couverture de l'ensemble des besoins
d'une entreprise. Non spécifique au BTP\\
\hline
Avantage&Logiciel éprouvé et adaptable à tous les métiers&Spécifique
&Spécifique et 20 ans d'expérience&Spécifique&Très complet\\
\hline
Inconvénients&Certaines fonctionnalités manquent&orienté PME
&Méconnu et donc fiabilité inconnue&Jeune
&Moins utilisé que SAP pour un niveau équivalent\\
\hline
\end{longtable}
\vfil
Nous avons évalué divers ERP orientés BTP au niveau de capacités nous
semblant essentielles ; tout comme pour le benchmarking des entreprises, il
convient de lire cette conclusion avec les points suivants à l'esprit :

\begin{itemize}
\item Nous ne pouvons aller plus en profondeur faute de temps.
\item Les prix sont mentionnés pour information mais nécessiteraient
d'établir un devis avec le fournisseur correspondant afin d'avoir une idée
plus précise.
\end{itemize}

\hfill\\

Nous avons examiné les pratiques de deux entreprises leader dans le domaine
du BTP : Vinci et Bouygue. Pour chacune, nous avons également relevé les
ERP utilisés :

\begin{itemize}
\item Vinci utilise Sage X3
\item Bouygue utilise SAP
\end{itemize}

%\vskip 6pt

Par ailleurs, nous avons évalué en dehors du contexte d'entreprise les possibilité des ERP suivants :

\begin{itemize}
\item SAP All-in-One
\item BRZ Pharos
\item PROGIB
\item Winlog
\item SAGE X3
\end{itemize}

\vskip 6pt

Afin d'effectuer un choix d'ERP, il nous faut à présent cerner plus
précisément les besoins de la Direction du Matériel de la société GSTP.

\pagebreak
