\subsection{Technologique}

\subsubsection{Informatisations des chantiers}

L’ensemble des chantiers seront informatisés.

\subsubsection{Uniformisation des logiciels}

Actuellement, chaque département a ses propres logiciels qui peuvent être diffèrent
les uns des autres. L’uniformisation facilitera les échanges de données et réduira
 les coûts de formation du personnel.

\subsubsection{Homogénéiser le SI}

Format standard pour les échanges de données, pour être sur d’une bonne communication
 et éviter les redondances. Donc modèle de données standard et uniformisation des logiciels.

\subsubsection{Outils mobiles par les chefs de chantier}

Pour un accès facile au système de n’importe quel endroit.

\subsubsection{Efficacité des moyens de communication}

Les moyens de communication actuels ne sont pas formalisés et lents. 
Une formalisation et une intégration des moyens de communications dans un SI 
permettrait une meilleure réactivité et une meilleure cohérence, donc un meilleur suivi.

\subsubsection{Forme du suivi}

Le suivi actuel est au format papier. Informatiser les fiches de suivi permettrait
 une uniformisation et un gain d’efficacité important. Le suivi informatique 
permet de transmettre les fiches directement.

\subsubsection{Planification informatisée}
La planification de l’utilisation du matériel doit être informatisée pour mieux 
la gérer et obtenir une aide logicielle.
