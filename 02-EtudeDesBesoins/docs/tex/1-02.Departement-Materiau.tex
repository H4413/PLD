\section{Département matériel}

\subsection{Organisation des services du dpt. matériel}


	\subsubsection{Gestion du matériel}

	Ce service est composée de 3 personnes. Il se charge de planifier
    l'affectation du matériel ainsi que les différentes zones de
    maintenance.
	
	\subsubsection{Gestion du parc matériel}
	
	Ce service est composée d'une personne qui s'occupe de la réception
    du matériel (Demande d'entrée de matériel au parc matériel central),
    de l'envoi du matériel (demande de sortie de matériel du parc matériel
    central), de la gestion d'entrée de matériel (entrée effective de
    matériel au garage après acceptation) et de la sortie de matériel
    (sortie effective de matériel du garage après acceptation).
		
	\subsubsection{Facturation matériel}
	Ce service est composé d'une personne qui a comme fonction principale de
    facturer correctement un chantier en fonction de la valorisation du
    matériel et du calcul du coût de la maintenance du matériel (coût de 
    maintenance, coût de pièce de rechange, coût de personnel).
	
\subsection{Procédures}
	
    Le département matériel a 2 procédures principales, les procédures de
    facturation et les procédures d'affectation.

\subsubsection{Procédure facturation}

	Le processus de facturation est réalisé par le service facturation
    matériel. La facture (périodicité mensuelle) de matériel est calculée a
    partir de la valorisation du matériel et du cout de maintenance:

    \begin{description}
	    \item [Valorisation du matériel:]\hfill\\
            La valorisation du matériel est calculée a partir:
		    \begin{itemize}
			    \item du pointage d'utilisation du matériel;
			    \item des avis valorisation frais de structure générés par
                    la gestion des matériels.
		    \end{itemize}	
    
    	\item [Coût de maintenance:]\hfill\\
    		Le calcul du coût de la maintenance est déterminée à partir:
		    \begin{itemize}
			    \item des avis valorisation personnes: c'est le coût de la
                    main d'\oe{}uvre  consomme par le matériel;
			    \item des avis valorisation pièces de rechange et pneus:
                    dépense en pièces de rechange et pneus.
		    \end{itemize}

    \end{description}

    {\sl Source:~{\ttfamily 
        GSTP/Ressources/Modele-de-l-existant/MCT-Facturer-chantier.doc}}

\subsection{Procédure affectation du matériel}

	Pour planifier l'affectation du matériel on distingue:
	\begin{itemize}
		\item L'affectation d'utilisation de matériel qui est faite en
            fonction des demandes des chantier pour le matériel;
		\item La demande de location de matériel, faites si aucun matériel
            est disponible pour satisfaire une demande. C'est le service de
            location qui fait la gestion de location;
		\item L'autorisation d'approvisionnement, validée à partir d'un
            calcul de besoin en fonctions de variation des stocks;
		\item La maintenance préventive du matériel effectuée après un
            certain temps d'utilisation.
	\end{itemize}

    {\sl Source:~{\ttfamily 
        GSTP/Ressources/Modele-de-l-existant/MCT-Planification.doc}}


\subsection{Liste des objets métier}

    \begin{itemize}
    	\item Facture
    	\item Chantier
    	\item Fournisseur
    	\item livraison
    	\item Affectation
    	\item Matériel
    	\item Pièce Rechange
    \end{itemize}

\subsection{Système d'information}

    \begin{description}
        \item [Matériel:]\el 
            Le département du matériel dispose de 3 ordinateurs (pour lancer
            l'application de facturation et de planning), qui sont reliés au
            serveur de l'entreprise et 2 imprimantes nécessaires pour
            éditer les factures de chaque chantier.

        \item [Logiciel:]
            Le département matériel utilise une application de facturation,
            et l'application de gestion de planning (pour le matériel).
    \end{description}

\subsection{Dysfonctionnements}


Non efficacité de Planification des affectations: lors d'une demande, il faut regarder si le materiel est disponible quelque part ou s'il sera disponible bientôt(2 ou 3 jours), pour ne pas louer ou d'acheter de matériel et faire du stock pas necessaire.

La demande de chantiers est faite sans délai d'avance: il faut un minimun de temps à l'avance pour trouver en stock disponible le matériel, sinon il y a le risque de faire des dépenses inutiles.

Disponibilité de matériel n'est pas souvent à jour: Il faut savoir à tout moment le date envoi et retour

Livraison pas directement sur le chantier
