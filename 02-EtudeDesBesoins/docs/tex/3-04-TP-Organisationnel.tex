\subsection{Organisationnel}

\subsubsection{Tableau de bords et {\sl reporting}}

Suivi de stock, livraison, maintenance, factures... de chaque chantier.

\subsubsection{Amélioration de la communication}

La communication entre les chantiers, les ateliers et les sièges doit être
plus rapide. Grâce à l’informatisation celle-ci peut être immédiate.

\subsubsection{Nécessité de planification}

Il n’y a pas de possibilité de planification de gestion du matériel. 
Il faut créer un système permettant de planifier l’utilisation du matériel.

\subsubsection{Changement de la gestion des magasins}

Suppression des magasins annexes, suppression d’un poste de magasinier et
renforcement du magasin central.

\subsubsection{Livraisons de pièces planifiées}
Les pièces sont livrées suivant un planning de façon hebdomadaire. 
Des outils de Business Intelligence permettront de déterminer les
itinéraires de livraison en fonction des commandes des chantiers et du
groupement des commandes.

\subsubsection{Gestion en temps réel}
Une gestion en temps réel de suivi du stock et des facturations
est une évolution 
d’importance stratégique pour GSTP. En effet cela permettra 
d’avoir une meilleure
 réactivité sur l'évolution du capital du chantier.

\subsubsection{Maintenance préventive et réduction des pannes}

Plus de maintenance préventive pour réduire les pannes et les coûts.
Nécessite des statistiques.

\subsubsection{Gestion du stock}
	Une meilleure gestion des stocks pour éviter le sur-stock et le stockage de
pièces inutilisées.

\subsubsection{Prévisions de consommation}
	Une meilleure prévision de la consommation des pièces de rechange
permettant une réduction de coût de commandes urgentes et des interventions
plus rapides.

\subsubsection{Référentiel de commande standard}
	Liste des pièces à commander en début de chantier. Les packs peuvent être
différents et sont principalement préétablis. Ces packs sont établis suivant
les statistiques.
