\subsection{Processus}
 
\subsubsection{Remplissage des fiches de suivi}

Le remplissage des fiches de suivi papier est remplacé par un remplissage de 
fiches de suivi informatisées. 
Cela permet un gain de temps et un meilleur formalisme au niveau processus.

\subsubsection{Processus d’urgence}

Réaliser des niveaux de priorité des processus, en créant un processus d’urgence
 (plus haute priorité), comme par exemple, une demande de matériel 
ou de maintenance d’urgence.

\subsubsection{Changement de la procédure de communication}

Le passage par un intervenant intermédiaire dans les communications rajoute une 
perte de temps inutile. 
Un processus de communication informatisé permettrait de raccourcir les délais.

\subsubsection{Mise en place de planification}

L’informatisation de la planification de l’utilisation du matériel doit faire 
l’objet d’un nouveau processus de planification. 

\subsubsection{Planification plus efficace}

Meilleure planification de la location du matériel par les chantiers et 
réduction de l’immobilisation en réduisant par exemple la sous utilisation
du matériel par un chantier.

\subsubsection{Commandes standard}

Mise en place d'un référentiel de commande standard pour raccourcir le temps de
commande et de préparation pour les produits commandés ensemble les plus
fréquemment.
