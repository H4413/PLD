\section{Synthèse globale de l'étude de l'existant}

De manière globale, on peut constater un manque de souplesse/d'agilité dû
aux points suivants :

\subsection{Système informatique}

\begin{description}
\item[Équipement matériel disparate]\el
Tous les départements sont équipés mais tous les chantiers ne le sont pas ;
cela induit des retards dus à la transmission de l'information, aux erreurs
de saisies dans le système, etc.  

\item[Équipement logiciel disparate]\el
Les différents départements utilisent différentes applications de gestion
développées en interne. Cela conduit à des logiciels peu ou pas documentés
et évolutifs. Les formats de données utilisés ne sont pas standards,
rendant difficile les interconnexions de logiciels de différents
départements. De manière générale, le SI n'est pas urbanisé : pas ouvert,
pas communiquant, pas évolutif.

\item[Pas de notion d'urgence clairement définie]\el
Les procédures définies fonctionnent mais aucune notion de hiérarchisation
n'apparaît nulle part concernant les demandes (affectation, achat, de
matériels ou pièces de rechange).  Il ne semble pas possible de qualifier
une demande de matériel ou de maintenance comme étant urgente ou critique.

\item[Pas de décision]\el
De manière générale, l'entreprise ne possède pas d'outil pour permettre la 
décision. C'est à dire de tableaux de bords et de {\sl reporting}.

\end{description}

\subsection{Fonctionnement général de l'entreprise}

Il y a une indétermination sur le placement général du département matériel 
vis à vis du reste de l'entreprise. On ne sait pas si le but de ce département
va être de générer du profit ou d'optimiser la gestion globale du parc sur le long
ou court terme. Le {\sl benchmarking} va nous permettre de comprendre le 
fonctionnement des acteurs majeurs du secteurs et ainsi trouver des solutions
pour améliorer le fonctionnement.

\subsection{Conclusion}

Au terme de l'étude de l'existant dans l'entreprise, on dispose d'un audit
portant sur le domaine de l'entreprise impacté le projet. Cet audit nous
permet de mieux cerner l'organisation du domaine et la manière dont il
fonctionne, afin de mieux réaliser le {\sl benchmark} dans la partie
suivante du dossier. Il permet également de pointer les faiblesses de
l'existant sur lesquelles nous seront amenés à concentrer nos efforts.
