\subsection{Comparaison des entreprises}

\begin{longtable}{|m{3cm}|m{4cm}|m{4cm}|m{4cm}|}
\hline
Critères&
%Eiffage&
Vinci&Bouygues\\
\endhead
\hline
Système de communication entre les départements – technologie utilisé, 
coût, vitesse de communication, qualité, etc...
&
%N.C
%&
ERP SAGE X3 - focus Vinci: la dématérialisation des documents,
le portail utilisateur, la Business Intelligence et l’accès web,
simplicité d’utilisation
&
N.C
\\
\hline
Système de planification du travail et des matériels 
&
%N.C
%&
N.C
&
SAP Business Information Warehouse
\\
\hline
Système de planification des stocks
&
%N.C
%&
N.C
&
N.C
\\
\hline
Comptabilité
&
%Spécifique
%&
Sage FRP Treasury
&
N.C
\\
\hline
Dimension économique (CA)
&
%13,23 milliards d'euros
%&
6,2 milliards d’euros
&
9,5 Milliards d'euros
\\
\hline
Facturation interne
&
%N.C
%&
Sage X3 Finances, la facturation à l’avancement ou la gestion des acomptes
&
SAP ECO\&O
\\
\hline
Intégration globale de la solution SI
&
%Unifiée en terme de sauvegardes
%\url{http://france.emc.com/collateral/customer-profiles/eiffage.pdf}
%&
ERP SAGE X3
&
SAP ECO\&O
\\
\hline
Hiérarchisation des tâches
&
%N.C
%&
N.C
&
N.C
\\
\hline
Deux entreprises ont été benchmarkées : Vinci et Bouygues. Il faut garder
plusieurs points à l'esprit durant la lecture de cette conclusion :

\begin{itemize}
\item GSTP, bien que de taille respectable (CA de 500 M\euro), n'est pas aussi
imposante que Vinci ou Bouygues.
\item Les deux entreprises benchmarkées ont choisi d'utiliser un ERP
global pour toutes leurs direction. Nous ne renouvellons que le SI de la
Direction du Matériel de GSTP
\item Nous n'avons pas les moyens d'effectuer un vrai benchmarking, qui
nécessite plusieurs choses :
    \begin{itemize}
    \item Du temps
    \item Partenariat avec l'entreprise cible
    \item Contact au sein de l'entreprise cible
    \item Visite de l'entreprise cible
    \end{itemize}
\end{itemize}


\end{longtable}
