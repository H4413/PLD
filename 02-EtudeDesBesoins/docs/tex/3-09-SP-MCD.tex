\section{Modèle conceptuel de données}

\subsection{Introduction}
Le MCD est disponible dans le fichier {\sl Gstp-mcd.pdf}.
Peu de modifications du MCD, simple ajout d'une classe {\emph commande standard}.

%Si le RQ est motivé il peut en faire un tableau....

\subsection{Ajout d'une classe Commande Standard}

La classe commande standard (STD\_CMD) représente des listes de pièces de 
rechange standard à commander à chaque début de chantier en fonction de son type.

\par{Description de l'entité}
\begin{description}
    \item[CODE-LISTE] Identifiant de la liste, TYPE : INT, LG : 10
    \item[NOM-LISTE] Nom de la liste de pièces, TYPE : STRING
    \item[DESC-LISTE] Description de la liste, TYPE : STRING
\end{description}

\par{Relations et cardinalité}
\begin{description}
    \item[COMM/LISTE] COMMANDE : Commande d'une liste standard (1,N)\el
        STD\_CMD : liste standard (0,N)
    \item[LISTE/PR] STD\_CMD : Liste des pièces de rechange (0,N)\el
        PIECE-RECHANGE :  Pièce de rechange.(0,N)
\end{description}

