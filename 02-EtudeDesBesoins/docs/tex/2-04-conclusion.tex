\subsection{Conclusion}

\subsubsection{Benchmarking des entreprises}

Deux entreprises ont été benchmarkées : Vinci et Bouygues. Il faut garder
plusieurs points à l'esprit durant la lecture de cette conclusion :

\begin{itemize}
\item GSTP, bien que de taille respectable (CA de 500 M\euro), n'est pas aussi
imposante que Vinci ou Bouygues.
\item Les deux entreprises benchmarkées ont choisi d'utiliser un ERP
global pour toutes leurs direction. Nous ne renouvellons que le SI de la
Direction du Matériel de GSTP
\item Nous n'avons pas les moyens d'effectuer un vrai benchmarking, qui
nécessite plusieurs choses :
    \begin{itemize}
    \item Du temps
    \item Partenariat avec l'entreprise cible
    \item Contact au sein de l'entreprise cible
    \item Visite de l'entreprise cible
    \end{itemize}
\end{itemize}


\subsubsection{Benchmarking des ERP}

Nous avons évalué divers ERP orientés BTP au niveau de capacités nous
semblant essentielles ; tout comme pour le benchmarking des entreprises, il
convient de lire cette conclusion avec les points suivants à l'esprit :

\begin{itemize}
\item Nous ne pouvons aller plus en profondeur faute de temps.
\item Les prix sont mentionnés pour information mais nécessiteraient
d'établir un devis avec le fournisseur correspondant afin d'avoir une idée
plus précise.
\end{itemize}

\hfill\\

Nous avons examiné les pratiques de deux entreprises leader dans le domaine
du BTP : Vinci et Bouygue. Pour chacune, nous avons également relevé les
ERP utilisés :

\begin{itemize}
\item Vinci utilise Sage X3
\item Bouygue utilise SAP
\end{itemize}

\vskip 6pt

Par ailleurs, nous avons évalué en dehors du contexte d'entreprise les possibilité des ERP suivants :

\begin{itemize}
\item SAP All-in-One
\item BRZ Pharos
\item PROGIB
\item Winlog
\item SAGE X3
\end{itemize}

\vskip 6pt

Afin d'effectuer un choix d'ERP, il nous faut à présent cerner plus
précisément les besoins de la Direction du Matériel de la société GSTP.
