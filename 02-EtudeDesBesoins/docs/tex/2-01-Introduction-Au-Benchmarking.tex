\vfil
\pagebreak
\section{Benchmarking}

\subsection{Introduction}

Les analyses effectuées sur l'étude de l'existant
de l'entreprise GSTP donne une
image globale des lacunes et des dysfonctionnements
de cette entreprise.

Pour déterminer l'état de l'art du marché tant en
terme d'ERP que de concurrence
nous allons procéder au benchmark en se basant sur
certains constats que l'on
a fait sur l'entreprise GSTP.

Voici ces constats:
\begin{itemize}
\item Communication lente entre le siège et les chantiers
\item Absence d'un format standard pour les échanges d'information
\item Applications internes indépendantes
\item Performances SI médiocres
\item Coûts d'achats et de stock trop importantes
\item Planificationis matériel médiocres
\item Immobilisation du matériel
\item Bas taux de disponibilité du stock
\item Hauts coûts de maintenance – trop de temps d'intervention,
beaucoup de pièces d'échange
\item Qualité en général médiocre
\end{itemize}

Après analyse du secteur du BTP nous observons 3 acteurs majeurs du secteurs :
Bouygues Construction, Vinci et Eiffage. De plus nous procéderons au benchmark
d'ERP générique comme SAP et d'ERP dédiés aux métiers du BTP.
\vfil
\pagebreak
