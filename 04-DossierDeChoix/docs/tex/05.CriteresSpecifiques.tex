\section{Justification des critères concernant la solution spécifique}

\subsection{Critères financiers}

B - Investissement total
L'investissement total est moins important que pour la mise en place d'un
système SAP mais reste conséquent.

B - Retour sur investissement
Le ROI est plus immédiat que lors de la mise en place d'un ERP car la
formation des utilisateurs est moins lourde. Toutefois sur le long terme la
solution ERP sera plus flexible.

C - Délai de retour sur investissement
Cf. ci-dessus.

\subsection{Mise en oeuvre}

D - Délai de mise en oeuvre
Le développement spécifique est potentiellement long. Cela dépend de la
réutilisation de composants spécifiques.

A - Délai d'adaptation au nouveau SI
Le SI spécifique est conçu au plus proche des besoins et attentes des
utilisateurs finaux. Ce délai est donc court.

A - Impact sur l'organisation
Les impacts sur l'organisation de GSTP sont très faibles car le SI
spécifique est conçu en fonction de l'organisation de GSTP.

	A - Des structures

	A - Des processus

	A - De la relation avec les partenaires

B - Risques de mise en oeuvre
Si le cahier des charges est respecté, les risques de mise en \oe{}uvre
sont minimes.

C - Autres

\subsection{Critères techniques et fonctionnels}

C - Facilité d'intégration dans le SI de l'entreprise
L'intégration avec les SI des autres départements (fruits de développements
spécifiques anciens) sera sans doute assez compliquée.

A - Adéquation aux besoins fonctionnels
Le SI spécifique étant développé au plus près des besoins du SI, ce critère
est facilement satisfait.

B - Qualités techniques

	C - Fiabilité, Sécurité
Le développement étant conduit spécifiquement, cela comporte plus de risque
(nous ne disposons de l'assurance qualité d'un éditeur ERP)

	D - Evolutivité, facilité de MàJ
Si le c\oe{}ur de métier du département vient à changer, le SI sera à
redévelopper.

	A - Facilité d'utilisation, ergonomie
Le SI est développé en ayant à l'esprit l'état de l'art en matière
d'Interface Utilisateur.

C - Autres

