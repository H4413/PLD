\section{Justification des critères concernant la solution spécifique}

\subsection{Critères financiers}

\begin{itemize}
\item[B -] Investissement total:\el
L'investissement total est moins important que pour la mise en place d'un
système SAP mais reste conséquent.

\item[B -] Retour sur investissement:\el
Le ROI est plus immédiat que lors de la mise en place d'un ERP car la
formation des utilisateurs est moins lourde. Toutefois sur le long terme la
solution ERP sera plus flexible.

\item[C -] Délai de retour sur investissement:\el
Cf. ci-dessus.
\end{itemize}

\subsection{Mise en œuvre}

\begin{itemize}
\item[D -] Délai de mise en œuvre:\el
Le développement spécifique est potentiellement long. Cela dépend de la
réutilisation de composants spécifiques.

\item[A -] Délai d'adaptation au nouveau SI:\el
Le SI spécifique est conçu au plus proche des besoins et attentes des
utilisateurs finaux. Ce délai est donc court.

\item[A -] Impact sur l'organisation:\el
Les impacts sur l'organisation de GSTP sont très faibles car le SI
spécifique est conçu en fonction de l'organisation de GSTP.

\begin{itemize}
	\item[A -] Des structures.

	\item[A -] Des processus.

	\item[A -] De la relation avec les partenaires.
\end{itemize}

\item[B -] Risques de mise en œuvre:\el
Si le cahier des charges est respecté, les risques de mise en \oe{}uvre
sont minimes.

\item[C -] Autres.
\end{itemize}

\subsection{Critères techniques et fonctionnels}

\begin{itemize}
\item[C -] Facilité d'intégration dans le SI de l'entreprise:\el
L'intégration avec les SI des autres départements (fruits de développements
spécifiques anciens) sera sans doute assez compliquée.

\item[A -] Adéquation aux besoins fonctionnels:\el
Le SI spécifique étant développé au plus près des besoins du SI, ce critère
est facilement satisfait.

\item[B -] Qualités techniques:\el
\begin{itemize}

	\item[C -] Fiabilité, Sécurité:\el
Le développement étant conduit spécifiquement, cela comporte plus de risque
(nous ne disposons de l'assurance qualité d'un éditeur ERP)

	\item[D -] Evolutivité, facilité de MàJ:\el
Si le c\oe{}ur de métier du département vient à changer, le SI sera à
redévelopper.

	\item[A -] Facilité d'utilisation, ergonomie:\el
Le SI est développé en ayant à l'esprit l'état de l'art en matière
d'Interface Utilisateur.
\end{itemize}

\item[C -] Autres.
\end{itemize}

