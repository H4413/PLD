\section{Solution SAP}

\subsection{Présentation Rapide}

La solution SAP va permettre de modifier le système d'information du département
matériel de la société GSTP en utilisant un ERP global : SAP.

La configuration de cet ERP est fait à partir de scénarion standard issu du package
Module Services Industries.

Voici la liste des scénarios utilisés par ce paramétrage.

Service Maintenance : \\
\begin{itemize}
    \item Réparation en atelier simplifiée. (275)
    \item Gestion de la maintenance et des garanties (274)
    \item Maintenance interne (193)
    \item Réparation en atelier (217)
    \item Gestion des réclamations et des retours (C38)
\end{itemize}

Service Matériel : \\
\begin{itemize}
    \item Gestion des équipements(444)
    \item Service avec prix fixe (200)
    \item Saisie des temps (211)
    \item Approvisionnement Matériel de remplacement (276)
    \item Reporting Interactif (CR2)
\end{itemize}
Service Achat : \\
\begin{itemize}
    \item Comptabilité fournisseurs (158)
    \item Grand Livre (156)
    \item Commande Pièce(R48) (dans package ECO)
\end{itemize}

\subsection{Avantages principaux}

La solution pourra facilement évoluer et permettre de s'adapter aux nouveaux besoins
fonctionnels de GSTP.


\subsection{Inconvénients majeurs}

L'investissement peut être surdimensionné pour équiper seulement un département
de l'entreprise.
