\section{Critères de choix}

Nous avons pris comme critères de choix les points suivants: 

\begin{description}
    \item  [Le prix:]\el
	Nous mettons en place ce critère, jugé comme indispensable.

    \item [Le ROI:]\el
	Les deux solutions peuvent générer des retours sur investissements
différents. Notamment dans le cadre des dépenses annexes et des différents
modes de gestions mis en place.

    \item [Les délais de mise en \oe{}uvre:]\el
	Le critère du {\sl time-to-market} est souvent l'un des éléments de
base pour décider.

    \item [Les facilités d'évolution:]\el
	Il faudra prendre en compte dans ce critère les possibilités d'évolutions 
de l'outil, à la fois pour développer de nouvelles fonctionnalités mais aussi 
les possibilités pour étendre la solution logicielle à d'autres départements de
GSTP.

    \item [L'intégration au sein de l'entreprise:]\el
	Le projet est-il sans risque pour l'entreprise? Va-t-il modifier les
fonctionnements de l'entreprise en profondeur? Le logiciel va-t-il s'intégrer
facilement avec les autres logiciels utilisés au sein de l'entreprise?

    \item [Le coût des formations utilisateurs et complexité de la solution]\el
	En utilisant un ERP standard calquant moins les processus exacts de GSTP ou 
en utilisant un ERP calqué sur les besoins de GSTP. De plus on jugera la
complexité de la solution lors d'une utilisation normale.

    \item [Les risques de la conduite du projet]\el
	Les équipes mises en place seront différentes pour les 2 projets.

\end{description}


	
