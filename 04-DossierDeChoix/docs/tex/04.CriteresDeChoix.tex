\section{Critères de Choix}

Nous mettons les critères de choix suivants : 

- Le prix

	Nous mettons en place ce critère, il est jugé comme indispensable.


- Le ROI

	Les deux solutions peuvent générer des retours sur investissements différents.
Notamment dans le cadre des dépenses annexes et des différents modes de gestions mis
en place.

- Les délais de mise en oeuvre

	Le critère du Time-To-Market est souvent l'un des éléments de base pour 
décider.

- Les facilités d'évolution

	Il faudra prendre en compte dans ce critère les possibilités d'évolutions 
de l'outil, à la fois pour développer de nouvelles fonctionnalités mais aussi 
les possibilités pour étendre la solution logicielle à d'autres départements de
GSTP

- L'intégration au sein de l'entreprise

	Le projet est-il sans risque pour l'entreprise? Va-t-il modifier les fonctionnements 
de l'entreprise en profondeur?	

- Le coût des formations utilisateurs et complexité de la solution

	En utilisant un ERP standard calquant moins les processus exacts de GSTP ou 
en utilisant un ERP calqué sur les besoins de GSTP. De plus on jugera la complexité
de la solution lors d'une utilisation normale.


- Les risques de la conduite du projet

	Les équipes mises en place seront différentes pour les 2 projets;



	
