\section{Solution spécifique}

\subsection{Présentation rapide}

La solution spécifique répond aux besoins de la direction matériel, de modifier
leur système d'information par le biais de développement d'applications spécifiques.

La nouvelle organisation utiliserait 4 applications distinctes mais communiquantes : 
\begin{itemize}
\item Une application matériel en charge de la planification de l'utilisation du 
matériel ainsi que de la communication. De plus, elle sera en charge de la gestion du 
patrimoine.
\item Une application achat, qui permet de gérer les commandes fournisseurs ainsi que
leurs listes.
\item Une application chantier, véritable interface entre le chantier et le siège 
social de GSTP; cet outil va s'occuper des besoins matériels des chantiers, de la planification et autres...
\item Uen application maintenance en charge d'établir les tâches de la maintenance ainsi que de les suivre 
jusqu'à leur réalisation.
\end{itemize}

\subsection{Avantages principaux}

Les principaux avantages de cette solution seront le fait qu'elle a été spécialement
conçue pour être utilisée par GSTP. Elle calquera les processus et les {\sl workflows}
utilisés par l'entreprise.

Le coût d'achat est relativement abordable.

Les coûts de formation des utilisateurs seront minimies car l'application sera 
adaptée pour GSTP. Les utilisateurs ne devront pas s'adapter à l'application.

\subsection{Inconvénients majeurs}

Nécessitant un développement plus important, l'application prendra plus temps
avant d'être opérationnelle. De plus, en cas de réorganisation
impactant le coeur de métier du département matériel, il sera obligatoire de modifier
en profondeur les applications. 

