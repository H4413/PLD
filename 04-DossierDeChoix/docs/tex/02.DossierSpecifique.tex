\section{Solution Spécifique}

\subsection{Présentation Rapide}

La solution spécifique répond aux besoins de la direction matériel, de modifier
leur système d'information par le biais de développement d'application spécifique.

La nouvelle organisation utiliserait 4 applications distinctes mais communiquantes : 
\begin{itemize}
\item Une application matériel en charge de la planification de l'utilisation du 
matériel ainsi que sa communication. De plus elle sera en charge de la gestion du 
patrimoine.
\item Une application achat, qui permet de gérer les commandes fournisseurs ainsi que
leurs listes
\item Une application chantier, véritable interface entre le chantier et le siège 
social de GSTP cet outil va so'ccuper des besoins matériel des chantiers de la planification et autres...
\item Uen application maintenance en charge d'établir les tâches de la maintenance ainsi que de les suivre 
jusqu'à leur réalisation.
\end{itemize}

\subsection{Avantages}

Les principaux avantages de cette solution seront le fait quelle a été spécialement
conçues pour être utilisée par GSTP. Elle calquera les processus et les workflows
utilisés par l'entreprise.

Le coût d'achat est reltaivement abordable.

Les coûts de formation aux utilisateurs seront minimies car l'application sera 
adaptée pour GSTP. Les utilisateurs ne devront pas s'adapter à l'application.

\subsection{Inconvénients}

Nécessitant un développement plus important, l'application nécessitera un temps
plus important avant d'être mise en production. De plus en cas de réorganisation
impactant le coeur de métier du département matériel, il sera nécessaire de modifier
en profondeur les applications. 
