\section{Choix d'une solution}

Le ROI est calcul� plus pr�cis�ment dans le dossier d'Evaluation Excel.

Nous sommes en pr�sence d'un facteur de co�t d'un rapport de 1,5 millions d'euros
� 4 millions d'euros pour la solution SAP. 

Le co�t important de la solution SAP est motiv� par le nombre d'utilisateur
important et le ticket d'entr�e qui est relativement cher. Cependant il est 
essentiel de comprendre que le co�t unitaire d�croit en fonction du nombre 
d'utilisateur. SAP modifie et structure en profondeur l'organisation d'une 
entreprise.

Il para�t surdimensionner pour �tre utiliser dans un seul d�partement. Le temps 
de ROI est de 10 ans, dur�e moyen de vie d'un ERP. Le risque est important pour 
ne pas atteindre ce point de retour, si on se limite � l'utilisation de SAP par
le d�partement mat�riel.

Pour la solution sp�cifique, elle est moins cher et va impacter faiblement 
l'organisation du d�partement mat�riel. Cette solution sera adapt� et n�cessite 
un faible temps d'adaptation de la part des utilisateurs. Le temps de ROI est 
plus rapide, il s'estime � 5 ans. N�anmoins cette solution ne sera que faiblement 
modulable en cas de modification ou de r�organisation de l'entreprise GSTP. 
Cette solution sera adapt� sur une p�riode moyenne.

De mani�re g�n�rale, la direction devra �tre prise en fonction de l'orientation
que souhaite prendre GSTP par rapport � leur organisation en g�n�ral. SAP doit �tre 
utilis� si l'entreprise d�cide de migrer compl�tement et totalement vers cet outil.
Elle disposera ainsi d'un syst�me coh�rent et facilement param�trable. De plus 
le co�t sera faible pour param�trer de nouvelles activit�s. En choisissant SAP, 
la direction ne pourra pas utiliser d'autres ERP que SAP. Elle se ferme donc 
des opportunit�s pour l'avenir.

Dans le cas contraire si la direction souhaite mettre en place une solution adapt�e
 uniquement au d�partement mat�riel. Il sera plus pertinant de mettre en place la
solution sp�cifique.





