\section{Justification des critères concernant la solution SAP}

\subsection{Critères financiers}

\begin{itemize}
\item [C -] Investissement total\el
	L'investissement pour utiliser SAP est assez conséquent,
    il faut être disposé à payer 4M€ pour un seul département.
	
\item [C -] Retour sur investissement
		
\item [D -] Délai de retour sur investissement\el
	Le délai de retour sur investissement est très long,
    car l'investissement total est conséquent.
\end{itemize}

\subsection{Mise en \oe{}uvre}

\begin{itemize}
\item [A -] Délai de mise en \oe{}uvre\el
	L'utilisation de la solution SAP est plus rapide que la solution
    spécifique, car le logiciel est déjà prêt, il suffit de le paramétrer.

\item [C -] Délai d'adaptation au nouveau SI\el
	Il est assez important, car les processus SAP peuvent être
    différents de ceux auxquels l'entreprise est habituée. Il faut
    donc prévoir un temps d'adaptation au nouveau vocabulaire pour
    les employés.

\item [B -] Impact sur l'organisation
    \begin{itemize}
	\item[B -] Des structures\el
		Il n'y a pas de modification profonde sur les structures
        de l'organisation.
		
	\item[C -] Des processus\el
		Les processus vont être calqués sur les processus standardisés
        par SAP. Ils seront donc différents de ceux qui l'entreprise
        est habitué.

	\item[A -] De la relation avec les partenaires\el
		Les processus standards SAP peuvent être aussi utilisés par
        les partenaires.
    \end{itemize}

\item[B -] Risques de mise en \oe{}uvre\el
	Il y a peu de risques dans la mise en \oe{}uvre de SAP, car les processus
    de mise en place de la solution sont fiables et prouvés par les acteurs
    du secteur. 
\end{itemize}
	

\subsection{Critères techniques et fonctionnels}

\begin{itemize}
\item[B -] Facilité d'intégration dans le SI de l'entreprise\el
	SAP possède plusieurs {\sl plug-in}, capables de s'intégrer à toute
    autre application, notamment les logiciels de comptabilité standards.
	
\item[B -] Adéquation aux besoins fonctionnels\el
	Le catalogue des processus standards est tellement riche qu'il
    couvre l'ensemble des besoins de l'entreprise.
	
\item[A -] Qualités techniques\el
    \begin{itemize}
	\item[A -] Fiabilité, Sécurité\el
		Qualité allemande. {\sl Deutsch Qualität}. Les acteurs du
        secteur l'ont prouvé pendant des années.
	
	\item[A -] Evolutivité, facilité de MàJ\el
		Il ne coûtera pas cher d'ajouter des nouveaux modules offerts par SAP. 

	\item[C -] Facilité d'utilisation, ergonomie\el
		Sur-dimensionné pour un seul département.
    \end{itemize}
\end{itemize}
