%7
\vfil
\pagebreak
\section{Annexes}

%ANNEXES :
%G – PLANS TYPES DES DOCUMENTS A LIVRER ( 2 à 3 pages )
%H - DESCRIPTION SUCCINCTE DES LOGICIELS A LIVRER :
%• reformulation des spécifications et/ou organigramme technique du produit ou système
%dans lequel s’inséreront les composants logiciels demandés
%I - DESCRIPTIF DES TACHES (document spécifique)

% 2
\subsection{Détail et plan des livrables}
%3-4 pages
%Liste et plans types des documents et des composants logiciels demandés (directement les
%annexes G et H si elles sont peu importantes
On retrouve ici les principales parties de chaque livrable.

\subsubsection{Dossier d'initialisation et PAQ}
		\begin{enumerate}
			\item Initialisation du projet:
				Dossier d'initialisation
					Plan type:
						\begin{enumerate}
							\item Objet la phase ; contexte ; positionnement
                                dans le cycle général du projet; liens avec
                                les autres phases, les autres projets.
							\item Résultats attendus (livrables à produire).
							\item Méthodes, modes opératoires, découpage en 
                                    phases.
							\item Pré-requis (documents, moyens, outils
                                    nécessaires).
							\item Planning des tâches; listes des tâches par
                                    ressources.
							\item Organisation de l'équipe.
							\item Plan de charge par ressource.
							\item Modalités de suivi d'avancement du projet.
							\item Modalités de validation et de recette.
							\item Amendement du plan d'assurance qualité.
							\item Plan de gestion des risques.
						\end{enumerate}
		\end{enumerate}
\subsubsection{Dossier d'expression des besoin}
		\begin{enumerate}
			\item Étude de l'existant:
				\begin{itemize}
					\item Rapport intermédiaire de synthèse de 5 pages utiles
				\end{itemize}
			\item Norme et \textsl{benchmark}:
				\begin{itemize} 
					\item Rapport intermédiaire de 4 pages utiles
				\end{itemize}
			\item Spécification du SI cible:
				\begin{itemize} 
					\item Rapport intermédiaire de modélisation
				\end{itemize}
		\end{enumerate}

\subsubsection{Dossier des solutions}
		\begin{enumerate}
			\item Spécification d'une solution spécifique:
				\begin{itemize}
					\item Rapport qui présente distinctement les dimensions
                            organisationnelle et informatique de la solution
				\end{itemize}
			\item Spécification d'une solution standard
				\begin{itemize} 
					\item Configuration des scénarios SAP sélectionnés(tableau)
					\item Matrice ARIS processus standard/organigramme GSTP 
				\end{itemize}
			\item Modélisation et configuration de la solution ERP:
				\begin{itemize} 
					\item Rapport de modélisation généré par ARIS
				\end{itemize}
		\end{enumerate}
		
\subsection{Dossier des choix}
		\begin{enumerate}
			\item Évaluation des solutions
				\begin{itemize}
					\item Dossier de Choix
				\end{itemize}
		\end{enumerate}

\subsubsection{Restitution}
		\begin{enumerate}
			\item Présentation avec support 
			\item Dossier Bilan
				\begin{enumerate}
					\item Planning général du projet
                        (avec positions début et fin de phase).
					\item Planning détaillé de la phase
                        (avec positions début et fin de phase).
					\item Tableau de bord de fin de phase en charge.
					\item Tableau de bord de fin de phase en délai.
					\item Tableau de bord de fin de phase de la production.
					\item Bilan de fonctionnement de l'organisation.
					\item Bilan du suivi des risques.
					\item Bilan du suivi de la qualité.
					\item Bilan financier.
					\item Bilan des contrats.
				\end{enumerate}
		\end{enumerate}
				
