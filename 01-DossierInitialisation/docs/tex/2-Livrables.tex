% 2
\section{Livrables}
%3-4 pages
%Liste et plans types des documents et des composants logiciels demandés (directement les
%annexes G et H si elles sont peu importantes

\subsection{Dossier d'initialisation et plan d'assurance qualité}

Ces documents vont assurer le lancement du projet et permettre d'organiser
la gestion des ressources le plus efficacement possible pour le dossier d'initialisation.
Le plan d'assurance qualité va permettre de définir le cadre du projet, c'est à dire les outils utilisés et les modalités de contrôle notamment. 

\subsection{Dossier d'expression des besoins}

Le dossier d'expression des besoins va comporter l'analyse de l'existant
de la société GSTP. Les équipes se formeront ensuite sur l'état de l'art
mis en place chez la concurrence (Vinci par exemple). L'ensemble des deux
analyses permettra de fournir les spécifications du système d'information cible
au travers de la modélisation faite par nos outils comme ARIS. 

\subsection{Dossier des solutions}

Le dossier des solutions va comprendre deux principaux volets :
\begin{itemize}
    \item La spécification d'une solution spécifique comprenant des rapports
        les dimensions organisationnelle et informatique de la solution;
    \item La spécification d'une solution standard utilisant l' ERP SAP et une
        modélisation avec l'outil ARIS.
\end{itemize}

\subsection{Dossier des choix}

Ce dossier sera un résumé de la partie précédente, il nous permettra de choisir
entre les deux solutions proposées, c'est à dire la solution spécifique et la
solution standard en fonction de critères de décision. 

\subsection{Restitution et dossier bilans}

Cette partie concernera à la fois la présentation finale prévue en semaine 10
ainsi que les dossiers bilans qui feront la synthèse du travail fourni par l'équipe.

\subsection{Récapitulatif des dates butoirs}

\begin{longtable}{|l|l|}
\hline
Livrables& Date\\
\endhead \hline
Dossier d'initialisation& 21/01/2011\\
\hline
Plan d'assurance qualité& 21/01/2011\\
\hline
Étude des besoins& 11/02/2011\\
\hline
Dossier des solutions& 18/02/2011\\
\hline
Dossier des choix& 25/02/2011\\
\hline
Restitution et bilans& 11/03/2011\\
\hline
Présentation&  11/03/2011\\
\hline
\end{longtable}

