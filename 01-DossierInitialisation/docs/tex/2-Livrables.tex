% 2
\section{livrables}
%3-4 pages
%Liste et plans types des documents et des composants logiciels demandés (directement les
%annexes G et H si elles sont peu importantes

\subsection{Dossier d'initialisation et Plan d'Assurance Qualité}

Ces documents vont assurer le lancement du projet et permettre d'organiser la gestion des ressources le plus efficacement possible pour le dossier d'initialisation.
Le Plan d'Assurance Qualité va permettre de définir le cadre du projet, c'est à dire les outils utilisés et les modalités de contrôle notamment. 

\subsection{Dossier d'Expression des Besoins}

Le dossier d'expression des besoins va comporter l'analyse de l'existant de la société GSTP. Ensuite les équipes se formeront sur l'état de l'art qui est mis en place chez la concurrence (exemple: Vinci). L'ensemble des deux analyses permettront de fournir les spécifications du Système d'Information Cible au travers de la modélisation faite par nos outils comme ARIS. 

\subsection{Dossier des solutions}

Le dossier des solutions va comprendre deux principaux volets :
\begin{itemize}
\item La spécification d'une solution spécifique comprenant des rapports les dimensions organisationelle et informatique de la solution
\item La spécification d'une solution standard utilisant le ERP SAP et une modélisation avec l'outil ARIS
\end{itemize}

\subsection{Dossier des choix}

Ce dossier sera un résumé de la partie précédente, il nous permettra de choisir entre les deux solutions proposées, c'est à dire la solution spécifique et la solution standard en fonction de critères de décision. 

\subsection{Restitution et Dossier Bilans}

Cette partie concernera à la fois la présentation finale prévue en semaine 10 ainsi que les dossiers bilans qui feront la syynthèse du travail fourni par l'équipe.
