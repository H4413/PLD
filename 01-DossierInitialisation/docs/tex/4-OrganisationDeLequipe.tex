%4
\section{Organisation de l'équipe}
%Définition des responsabilités et des rôles de chaque membre de l’équipe
%Histogramme des charges par personnes ( résultant du planning )

\subsection{Organisation globale}

L'équipe d'intervention pour la réalisation de l'étude préalable pour le compte de la société XXXXX est composée de : 

\begin{itemize}
\item {\bf Quentin VILLERS} est nommé {\bf Chef de Projet} {\it (CdP)} pour son expérience sur des projets longue durée acquis au sein des diverses missions à la Junior-Entreprise de l'INSA de Lyon, ETIC Insa Technologies.
\item {\bf Raphaël LIZE} est nommé {\bf Responsable Qualité} {\it (RQ)} pour sa maîtrise des outils. Il arrivera à transmettre son savoir à l'ensemble du groupe, de plus rigoureux, Raphaël sera déniché les points insuffissants des dossiers. 
\item {\bf Hugo PASTORE DE CRISTOFARO} est nommé {\bf Responsable Communication} pour son approche commercial et la qualité de ses présentations. 
\item {\bf Karen ABANTO} est nommé {\bf Responsable Recherche}, à l'affût des dernières technologies et son parcours universitaire lui permettront de mener à bien ses missions de veille.
\item {\bf Victor BORGES FERREIRA GOMES} est nommé {\bf Responsable de l'Analyse}, au cours de son parcours il a su montrer ses capacités pour choisir les solutions les plus pertinentes.
\item {\bf Clément GEIGER} est nommé {\bf Secrétaire Général} du groupe pour sa maitrise des subtilités de la langue. Il aidera le responsable qualité pour les relectures de documents.  
\end{itemize}

\subsection{Histogramme de la charge par collaborateur}

\begin{longtable}{|r|l|l|l|l|l|l|l|}
\hline
&Clément&Hugo&Karen&Quentin&Raphaël&Victor&Charge/Semaine\\
\endhead \hline
S01&3,5&3,5&3,5&6,5&3,5&3,5&24\\
\hline
S02&4,5&4,5&7,5&5,5&6,5&3,5&32\\
\hline
S03&3&5&5&4,5&4,5&5&27\\
\hline
S04&3,5&3,5&4,5&5&3,5&3,5&23,5\\
\hline
S05&3,5&3,5&4,5&4,5&5&3,5&24,5\\
\hline
S06&3,5&4,5&4,5&3,5&4&3,5&23,5\\
\hline
S07&3,5&4,5&4,5&5,5&5,5&3,5&27\\
\hline
S08&5,5&6,5&4,5&6,5&4&3,5&30,5\\
\hline
S09&2&3&2&9&7&2&25\\
\hline
S10&4&4&4&4&4&4&24\\
\hline
Charge/Pers.&36,5&42,5&44,5&54,5&47,5&35,5&\\
\hline
\end{longtable}


\subsection{Outils de suivi}

Le chef de projet mettra en place des outils de suivi pour mettre gérer l'ensemble de son équipe. 
Ces indicateurs seront : 
\begin{itemize}
\item un indicateur de retard
\item une évaluation du moral (note sur 4)
\item une autoévalution (critères (-) , 0 ou (+) )
\item un suivi de la charge prévue
\item un suivi de la charge effective
\end{itemize}

Au-delà de ses indicateurs chiffrés, un rapport hebdomadaire sera produit comprennant le suivi de la charge effective, un suivi des risques, des livrables et un bilan moral de l'activité de la semaine.
