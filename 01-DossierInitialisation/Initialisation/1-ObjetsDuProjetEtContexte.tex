\section{Objet du projet et contexte}
%1-2 pages
%L’objet du projet
%Le contexte général du projet ; son positionnement éventuel dans un projet plus vaste ; synthèse
%des phases antérieures si il y a lieu.
%Son positionnement dans le cycle de vie général du développement des système d’information (
%identification du type de phase à laquelle correspond le projet ; ex : étude préalable,
%spécification d’interface, étude d’architecture technique, réalisation, test, ....)

\subsection{Objet du projet GSTP}
Le projet est un étude préalable de la conception et de l'automatisation du système d'information dans l'entreprise GSTP qui est spécialisée dans les activités de terrassement et génie civil.\\

Le but est de déterminer le périmètre du projet SI et sa faisabilité, c’est-à-dire de définir ce qui sera inclu dans le projet, ce qui ne le sera pas et si le projet doit bien être lancé.\\

D’une part, on estime si les bénéfices attendus seront en proportion des investissements engagés et du coût prévisionnel du projet SI.
D’autre part, l’étude préalable détermine également si l’entreprise GSTP est bien en mesure de mener ce projet à son terme. On cherche en particulier à savoir si elle dispose des compétences, des ressources et des fonds nécessaires.\\

Parler d'urbanisation du SI de GSTP ! C'est l'objet du projet après tout.


\subsection{Contexte général du projet}

Monde du bâtiment et travaux publiques (GSTP = Génie et Services dans Travaux Publiques)

Entreprise déjà existante et bien structurée (cf. organigramme, développer sur la structure et ce que ça implique pour le projet). Eventuellement expliquer rapidement le fonctionnement de l'organigramme de la société (rôle de chaque entité). L'entreprise est assez importante (citer quelques chiffres).

Division siège/chantier

siège : entité unique où sont regroupés les divers comités de direction de l'entreprise (DG, DRH, DFC, etc. cf. sujet)

chantiers : multiples, répartis dans un rayon de 500 km autour du siège. Les chantiers jouissent d'une certaine autonomie de fonctionnement et financière par rapport au siège. Les chantiers sont planifiés par la DTEM au siège de la société. Le chantier est décrit par une multitude d'attributs du point gestion (code, nom, localisation, etc.)

L'équipement des chantiers en matériel informatique est disparate (un tiers des chantiers est équipé), mais l'ensemble sera équipé sur un horizon de dix mois. 

\subsection{Présentation de l'entreprise}

\subsubsection{Secteur d'activité}
\subsubsection{Organigramme}
\subsubsection{Equipement informatique}
\subsubsection{Organisation structurelle}


\subsection{Identification des secteurs de l'entreprise impactés par le projet}

\subsubsection{Activités de l'entreprise}
\subsubsection{Directions et services}
\subsubsection{Processus et procédures stratégiques}
