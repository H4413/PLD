\section{Objet du projet et contexte}
%1-2 pages
%L’objet du projet
%Le contexte général du projet ; son positionnement éventuel dans un projet plus vaste ; synthèse
%des phases antérieures si il y a lieu.
%Son positionnement dans le cycle de vie général du développement des système d’information (
%identification du type de phase à laquelle correspond le projet ; ex : étude préalable,
%spécification d’interface, étude d’architecture technique, réalisation, test, ....)

\subsection{Objet du projet GSTP}
Le projet est un étude préalable de la conception et de l'automatisation du système d'information dans l'entreprise GSTP qui est spécialisée dans les activités de terrassement et génie civil.\\

Le but est de déterminer le périmètre du projet SI et sa faisabilité, c’est-à-dire de définir ce qui sera inclus dans le projet, ce qui ne le sera pas et si le projet doit bien être lancé.\\

D’une part, on estime si les bénéfices attendus seront en proportion des investissements engagés et du coût prévisionnel du projet SI.
D’autre part, l’étude préalable détermine également si l’entreprise GSTP est bien en mesure de mener ce projet à son terme. On cherche en particulier à savoir si elle dispose des compétences, des ressources et des fonds nécessaires.\\


\subsection{Contexte général du projet}

