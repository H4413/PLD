%PARTIE SUR LA GESTION DES RISQUES (5)
\section{Gestion des Risques}

\subsection{Non maitrîse des outils et perte en réactivité - HUMAIN}
\subsubsection{Analyse des causes}

Les technologies utilisées pour le travail collaboration lors de ce projet longue durée ne sont pas maitrisées par l'ensemble du groupe.
On peut citer notamment 2 personnes qui n'ont pas encore utiliser Latex comme outil de rédaction et 3 personnes qui n'avaient jamais utlisé Git comme outil de versionnement. //
Le risque est {\bf probable}.

\subsubsection{Analyse des conséquences}

Le degrée des formations des acteurs n'étant pas le même sur les technologies utilisées, il sera ainsi plus difficile dans un premier temps d'homogénéiser la charge. Les collaborateurs les plus formés devront effectuer un travail de relecture et d'accompagnement important. 

L'impact peut être un manque de cohésion entre les différents collaborateurs d'un point de vue des méthodes et de la charge de travail.

\subsubsection{Actions de surveillance}

\begin{enumerate}
\item {\bf Suivre} l'évolution de la formation sur les premières séances \\ Responsable {\bf CdP}
\item {\bf Mettre en place des indicateurs} pour vérifier l'état de formation des participants \\ Responsable {\bf CdP}
\end{enumerate}

\subsubsection{Action d'intervention}

\begin{enumerate}
\item {\bf Organiser} une formation sur les outils \\ Responsable {\bf CdP} et {\bf RQ}
\item {\bf Documenter} les différentes procédures et mettre en place des walkthroughs \\ Responsable {\bf RQ}
\item {\bf Homogéneiser} la charge en répatissant les tâches en fonction des compétences \\ Responsable {\bf RQ}
\item {\bf Former} les collaborateurs sur les différents outils de travail \\ Responsable {\bf RQ}
\item {\bf Modifier} les standards utilisées pour convenir au plus grand nombre (à n'utiliser qu'en dernier recours) \\ Responsable {\bf CdP}
\end{enumerate}

%PARTIE SUR LA NON MAITRISE TECHNOLOGIQUE
\subsection{Dériver par rapport à nos objectifs initiaux - HUMAIN}
\subsubsection{Analyse des causes}

L'équipe n'a pas encore fait d'étude préalable par rapport à un système d'information. Elle ne sait pas où elle va, malgré l'éventail documentaire fourni dans la formation. 
La possibilité est relativement {\bf importante}.

\subsubsection{Analyse des conséquences}

L'équipe peut perdre du temps en cherchant dans les mauvaises directions. Ainsi le rendement perdra en efficacité et conduira à de la sous-qualité car les parties utiles seront en quelque sorte baclée. 

\subsubsection{Actions de surveillance}

\begin{enumerate}
\item {\bf Suivre} l'avancement des différents livrables par rapport au gantt initial fourni avec ce Dossier d'initialisation \\ Responsable {\bf CdP}
\end{enumerate}

\subsubsection{Action d'intervention}

\begin{enumerate}
\item {\bf Organiser} des réunions de suivi et de briefing à chaque début de séance \\ Responsable {\bf CdP}
\item {\bf Organiser} des réunions de synthèse à la fin de chaque séance pour enchainer de manière efficace les séances de travail \\ Responsable {\bf CdP}
\end{enumerate}
