\section{Liste des objets métier à développer}

La liste des objets métiers est présentée sous la forme d'un tableau à deux
entrées. De cette façon, on peut plus facilement voir quels sont les objets
métiers réutilisés plusieurs fois par différentes applications.

%\begin{table}[h]
%\centering
%\caption{Liste des objets métier à développer}
\begin{longtable}{|c|c|c|c|c|}
\hline
\backslashbox{Listes Objets Métiers}{Applications}&Achats& Matériel&Maintenance&Chantier\\
\endhead
\hline
Affectation&&$\surd$&&$\surd$\\
\hline
Atelier&&$\surd$&$\surd$&\\
\hline
Chantier&&$\surd$&$\surd$&\\
\hline
Commande&$\surd$&$\surd$&$\surd$&$\surd$\\
\hline
Facture&&$\surd$&&$\surd$\\
\hline
Fonction&&&$\surd$&\\
\hline
Catalogue fournisseurs&$\surd$&&&\\
\hline
Fournisseur&$\surd$&&&\\
\hline
Gamme maintenance&&&$\surd$&\\
\hline
Livraison&$\surd$&&&\\
\hline
Matériel&&$\surd$&$\surd$&$\surd$\\
\hline
Opération maintenance&&&$\surd$&$\surd$\\
\hline
Période&&$\surd$&$\surd$&$\surd$\\
\hline
Piece rechange&$\surd$&&$\surd$&$\surd$\\
\hline
Planification d'affectation&&$\surd$&$\surd$&$\surd$\\
\hline
Planification de maintenance&&$\surd$&$\surd$&\\
\hline
Poste fonctionnel&$\surd$&$\surd$&$\surd$&$\surd$\\
\hline
Prestation&$\surd$&&&\\
\hline
Produit acheté&$\surd$&&&\\
\hline
Regroupement de produits&&$\surd$&&\\
\hline
Type de matériel&&$\surd$&&\\
\hline
Type d'opération de maintenance&&&$\surd$&\\
\hline
Type de période&$\surd$&$\surd$&$\surd$&$\surd$\\
\hline
Type de prestation&$\surd$&&&\\
\hline
Type de regroupement de produit&&$\surd$&&\\
\hline
\end{longtable}
%\end{table}
